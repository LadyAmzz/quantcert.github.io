%% Generated by Sphinx.
\def\sphinxdocclass{report}
\documentclass[letterpaper,10pt,english]{sphinxmanual}
\ifdefined\pdfpxdimen
   \let\sphinxpxdimen\pdfpxdimen\else\newdimen\sphinxpxdimen
\fi \sphinxpxdimen=.75bp\relax

\PassOptionsToPackage{warn}{textcomp}
\usepackage[utf8]{inputenc}
\ifdefined\DeclareUnicodeCharacter
 \ifdefined\DeclareUnicodeCharacterAsOptional
  \DeclareUnicodeCharacter{"00A0}{\nobreakspace}
  \DeclareUnicodeCharacter{"2500}{\sphinxunichar{2500}}
  \DeclareUnicodeCharacter{"2502}{\sphinxunichar{2502}}
  \DeclareUnicodeCharacter{"2514}{\sphinxunichar{2514}}
  \DeclareUnicodeCharacter{"251C}{\sphinxunichar{251C}}
  \DeclareUnicodeCharacter{"2572}{\textbackslash}
 \else
  \DeclareUnicodeCharacter{00A0}{\nobreakspace}
  \DeclareUnicodeCharacter{2500}{\sphinxunichar{2500}}
  \DeclareUnicodeCharacter{2502}{\sphinxunichar{2502}}
  \DeclareUnicodeCharacter{2514}{\sphinxunichar{2514}}
  \DeclareUnicodeCharacter{251C}{\sphinxunichar{251C}}
  \DeclareUnicodeCharacter{2572}{\textbackslash}
 \fi
\fi
\usepackage{cmap}
\usepackage[T1]{fontenc}
\usepackage{amsmath,amssymb,amstext}
\usepackage{babel}
\usepackage{times}
\usepackage[Bjarne]{fncychap}
\usepackage{sphinx}

\usepackage{geometry}

% Include hyperref last.
\usepackage{hyperref}
% Fix anchor placement for figures with captions.
\usepackage{hypcap}% it must be loaded after hyperref.
% Set up styles of URL: it should be placed after hyperref.
\urlstyle{same}

\addto\captionsenglish{\renewcommand{\figurename}{Fig.}}
\addto\captionsenglish{\renewcommand{\tablename}{Table}}
\addto\captionsenglish{\renewcommand{\literalblockname}{Listing}}

\addto\captionsenglish{\renewcommand{\literalblockcontinuedname}{continued from previous page}}
\addto\captionsenglish{\renewcommand{\literalblockcontinuesname}{continues on next page}}

\addto\extrasenglish{\def\pageautorefname{page}}

\setcounter{tocdepth}{0}


% One-column index
\makeatletter
\renewenvironment{theindex}{
  \chapter*{\indexname}
  \markboth{\MakeUppercase\indexname}{\MakeUppercase\indexname}
  \setlength{\parskip}{0.1em}
  \relax
  \let\item\@idxitem
}{}
\makeatother
\renewcommand{\ttdefault}{txtt}


\title{Documentation of Mermin-evaluation}
\date{Dec 19, 2019}
\release{1.0}
\author{Henri de Boutray}
\newcommand{\sphinxlogo}{\vbox{}}
\renewcommand{\releasename}{Release}
\makeindex

\begin{document}

\maketitle
\sphinxtableofcontents
\phantomsection\label{\detokenize{index::doc}}


This is the reference manual for a SageMath package \sphinxcode{\sphinxupquote{grover\_ent}}.

To use this module, you need to import it:

\fvset{hllines={, ,}}%
\begin{sphinxVerbatim}[commandchars=\\\{\},formatcom=\footnotesize]
\PYG{k+kn}{from} \PYG{n+nn}{grover\PYGZus{}ent} \PYG{k}{import} \PYG{o}{*}
\end{sphinxVerbatim}

This package is used and documented in project available \sphinxurl{https://quantcert.github.io/Grover\_ent}


\chapter{Grover entanglement}
\label{\detokenize{index:grover-entanglement}}

\section{Grover main}
\label{\detokenize{grover:module-grover}}\label{\detokenize{grover:grover-main}}\label{\detokenize{grover::doc}}\index{grover (module)}
This module is aimed to provide the user means to study the Grover algorithm and
especially how the entanglement changes during its execution using the Mermin 
evaluation.
\index{diffusion\_artificial() (in module grover)}

\begin{fulllineitems}
\phantomsection\label{\detokenize{grover:grover.diffusion_artificial}}\pysiglinewithargsret{\sphinxcode{\sphinxupquote{grover.}}\sphinxbfcode{\sphinxupquote{diffusion\_artificial}}}{\emph{V}}{}
Performs the inversion about the mean for \sphinxcode{\sphinxupquote{V}}.
\begin{quote}\begin{description}
\item[{Parameters}] \leavevmode
\sphinxstyleliteralstrong{\sphinxupquote{V}} (\sphinxstyleliteralemphasis{\sphinxupquote{vector}}) \textendash{} The running state in Grover’s algorithm .

\item[{Returns}] \leavevmode
vector \textendash{} \sphinxcode{\sphinxupquote{V}} Inverted about the mean.

\end{description}\end{quote}

\end{fulllineitems}

\index{grover() (in module grover)}

\begin{fulllineitems}
\phantomsection\label{\detokenize{grover:grover.grover}}\pysiglinewithargsret{\sphinxcode{\sphinxupquote{grover.}}\sphinxbfcode{\sphinxupquote{grover}}}{\emph{target\_state\_vector}, \emph{verbose=False}, \emph{file\_name=None}, \emph{use\_precomputed=False}, \emph{artificial=False}}{}
Prints in terminal or in file \sphinxcode{\sphinxupquote{file\_name}} the Mermin evaluation of each step 
of the Grover algorithm.
\begin{description}
\item[{Example:}] \leavevmode
\fvset{hllines={, ,}}%
\begin{sphinxVerbatim}[commandchars=\\\{\},formatcom=\footnotesize]
\PYG{g+gp}{\PYGZgt{}\PYGZgt{}\PYGZgt{} }\PYG{n}{grover}\PYG{p}{(}\PYG{n}{target\PYGZus{}state\PYGZus{}ket\PYGZus{}string\PYGZus{}to\PYGZus{}vector}\PYG{p}{(}\PYG{l+s+s2}{\PYGZdq{}}\PYG{l+s+s2}{0000}\PYG{l+s+s2}{\PYGZdq{}}\PYG{p}{)}\PYG{p}{)}
\PYG{g+go}{[0.168628057515893, 1.32148634347176, 1.01189404012546, 0.469906068135870]}
\end{sphinxVerbatim}

\end{description}
\begin{quote}\begin{description}
\item[{Parameters}] \leavevmode\begin{itemize}
\item {} 
\sphinxstyleliteralstrong{\sphinxupquote{target\_state\_vector}} (\sphinxstyleliteralemphasis{\sphinxupquote{vector}}\sphinxstyleliteralemphasis{\sphinxupquote{{[}}}\sphinxhref{https://docs.python.org/3/library/functions.html\#int}{\sphinxstyleliteralemphasis{\sphinxupquote{int}}}\sphinxstyleliteralemphasis{\sphinxupquote{{]}}}) \textendash{} State searched by Grover’s algorithm 
(only single item searches are supported for now).

\item {} 
\sphinxstyleliteralstrong{\sphinxupquote{verbose}} (\sphinxhref{https://docs.python.org/3/library/functions.html\#bool}{\sphinxstyleliteralemphasis{\sphinxupquote{bool}}}) \textendash{} If \sphinxcode{\sphinxupquote{verbose}} then extra run information will be 
displayed in terminal.

\item {} 
\sphinxstyleliteralstrong{\sphinxupquote{file\_name}} (\sphinxhref{https://docs.python.org/3/library/stdtypes.html\#str}{\sphinxstyleliteralemphasis{\sphinxupquote{str}}}) \textendash{} File name for the registration of the Mermin evaluation 
for each step of the algorithm, in csv format.

\item {} 
\sphinxstyleliteralstrong{\sphinxupquote{use\_precomputed}} (\sphinxhref{https://docs.python.org/3/library/functions.html\#bool}{\sphinxstyleliteralemphasis{\sphinxupquote{bool}}}) \textendash{} for some states, the optimal Mermin polynomial 
has been precomputed, use this option to speed up the overall computation.

\item {} 
\sphinxstyleliteralstrong{\sphinxupquote{artificial}} (\sphinxhref{https://docs.python.org/3/library/functions.html\#bool}{\sphinxstyleliteralemphasis{\sphinxupquote{bool}}}) \textendash{} Due to technological limits, it is not always possible 
to compute the states of Grover’s algorithm in a realistic simulator. This 
parameters allows the user to use a non realistic simulator which is way 
quicker.

\end{itemize}

\item[{Returns}] \leavevmode
any \textendash{} Result of this function depends on file\_name. If a file name
is given \sphinxcode{\sphinxupquote{qft\_main}} returns None, otherwise, it returns an array with the 
evaluation values.

\end{description}\end{quote}

\end{fulllineitems}

\index{grover\_artifical() (in module grover)}

\begin{fulllineitems}
\phantomsection\label{\detokenize{grover:grover.grover_artifical}}\pysiglinewithargsret{\sphinxcode{\sphinxupquote{grover.}}\sphinxbfcode{\sphinxupquote{grover\_artifical}}}{\emph{target\_state\_vector}}{}
To accelerate the computation of the states, the alternative to 
\sphinxcode{\sphinxupquote{grover\_vanilla}} is to work directly on the vectors. INdeed, we know what 
the effects of each steps are, so we don’t need to apply the gates we can 
simply change the vectors accordingly. For big dimensions, this is a way more 
efficient method.
\begin{quote}\begin{description}
\item[{Parameters}] \leavevmode
\sphinxstyleliteralstrong{\sphinxupquote{target\_state\_vector}} (\sphinxstyleliteralemphasis{\sphinxupquote{vector}}\sphinxstyleliteralemphasis{\sphinxupquote{{[}}}\sphinxhref{https://docs.python.org/3/library/functions.html\#int}{\sphinxstyleliteralemphasis{\sphinxupquote{int}}}\sphinxstyleliteralemphasis{\sphinxupquote{{]}}}) \textendash{} State searched by Grover’s algorithm 
(only single item searches are supported for now).

\item[{Returns}] \leavevmode
list{[}vector{]} \textendash{} List of states after each application of the 
diffusion operator (as well as the initial state).

\end{description}\end{quote}

\end{fulllineitems}

\index{grover\_evaluate() (in module grover)}

\begin{fulllineitems}
\phantomsection\label{\detokenize{grover:grover.grover_evaluate}}\pysiglinewithargsret{\sphinxcode{\sphinxupquote{grover.}}\sphinxbfcode{\sphinxupquote{grover\_evaluate}}}{\emph{end\_loop\_states}, \emph{M\_opt}, \emph{file\_name}}{}
Uses the previously found optimal mermin operator to evaluate the entanglement 
for each state in \sphinxcode{\sphinxupquote{end\_loop\_states}}.
\begin{quote}\begin{description}
\item[{Parameters}] \leavevmode\begin{itemize}
\item {} 
\sphinxstyleliteralstrong{\sphinxupquote{end\_loop\_states}} (\sphinxhref{https://docs.python.org/3/library/stdtypes.html\#list}{\sphinxstyleliteralemphasis{\sphinxupquote{list}}}\sphinxstyleliteralemphasis{\sphinxupquote{{[}}}\sphinxstyleliteralemphasis{\sphinxupquote{vector}}\sphinxstyleliteralemphasis{\sphinxupquote{{]}}}) \textendash{} The states at the end of each loop

\item {} 
\sphinxstyleliteralstrong{\sphinxupquote{M\_opt}} (\sphinxstyleliteralemphasis{\sphinxupquote{matrix}}) \textendash{} The optimal Mermin operator.

\item {} 
\sphinxstyleliteralstrong{\sphinxupquote{file\_name}} (\sphinxhref{https://docs.python.org/3/library/stdtypes.html\#str}{\sphinxstyleliteralemphasis{\sphinxupquote{str}}}) \textendash{} File name for the registration of the Mermin evaluation 
for each step of the algorithm, in csv format.

\end{itemize}

\item[{Returns}] \leavevmode
any \textendash{} Result of this function depends on file\_name. If a file name
is given \sphinxcode{\sphinxupquote{qft\_main}} returns None, otherwise, it returns an array with the 
evaluation values.

\end{description}\end{quote}

\end{fulllineitems}

\index{grover\_layers\_kopt() (in module grover)}

\begin{fulllineitems}
\phantomsection\label{\detokenize{grover:grover.grover_layers_kopt}}\pysiglinewithargsret{\sphinxcode{\sphinxupquote{grover.}}\sphinxbfcode{\sphinxupquote{grover\_layers\_kopt}}}{\emph{target\_state\_vector}}{}
Computes the circuit for Grover’s algorithm using the circuit format used for 
the \sphinxcode{\sphinxupquote{run}} function from \sphinxcode{\sphinxupquote{run\_circuit.sage}}.
\begin{quote}\begin{description}
\item[{Parameters}] \leavevmode
\sphinxstyleliteralstrong{\sphinxupquote{target\_state\_vector}} (\sphinxstyleliteralemphasis{\sphinxupquote{vector}}\sphinxstyleliteralemphasis{\sphinxupquote{{[}}}\sphinxhref{https://docs.python.org/3/library/functions.html\#int}{\sphinxstyleliteralemphasis{\sphinxupquote{int}}}\sphinxstyleliteralemphasis{\sphinxupquote{{]}}}) \textendash{} State searched by Grover’s algorithm 
(only single item searches are supported for now).

\item[{Returns}] \leavevmode
(list{[}list{[}matrix{]}{]},int) \textendash{} Circuit for Grover’s algorithm and 
optimal value found for \sphinxcode{\sphinxupquote{k\_opt}}.

\end{description}\end{quote}

\end{fulllineitems}

\index{grover\_optimize() (in module grover)}

\begin{fulllineitems}
\phantomsection\label{\detokenize{grover:grover.grover_optimize}}\pysiglinewithargsret{\sphinxcode{\sphinxupquote{grover.}}\sphinxbfcode{\sphinxupquote{grover\_optimize}}}{\emph{target\_state\_vector}, \emph{use\_precomputed=False}, \emph{verbose=False}}{}
Computes the optimal Mermin operator to evaluate the entanglement in the
Grover algorithm
\begin{quote}\begin{description}
\item[{Parameters}] \leavevmode\begin{itemize}
\item {} 
\sphinxstyleliteralstrong{\sphinxupquote{target\_state\_vector}} (\sphinxstyleliteralemphasis{\sphinxupquote{vector}}\sphinxstyleliteralemphasis{\sphinxupquote{{[}}}\sphinxhref{https://docs.python.org/3/library/functions.html\#int}{\sphinxstyleliteralemphasis{\sphinxupquote{int}}}\sphinxstyleliteralemphasis{\sphinxupquote{{]}}}) \textendash{} State searched by Grover’s algorithm 
(only single item searches are supported for now).

\item {} 
\sphinxstyleliteralstrong{\sphinxupquote{use\_precomputed}} (\sphinxhref{https://docs.python.org/3/library/functions.html\#bool}{\sphinxstyleliteralemphasis{\sphinxupquote{bool}}}) \textendash{} For some states, the optimal Mermin polynomial 
has been precomputed, use this option to speed up the overall computation.

\item {} 
\sphinxstyleliteralstrong{\sphinxupquote{verbose}} (\sphinxhref{https://docs.python.org/3/library/functions.html\#bool}{\sphinxstyleliteralemphasis{\sphinxupquote{bool}}}) \textendash{} If \(verbose\) then extra run information will be displayed 
in terminal.

\end{itemize}

\item[{Returns}] \leavevmode
matrix \textendash{} The optimal Mermin operator.

\end{description}\end{quote}

\end{fulllineitems}

\index{grover\_run() (in module grover)}

\begin{fulllineitems}
\phantomsection\label{\detokenize{grover:grover.grover_run}}\pysiglinewithargsret{\sphinxcode{\sphinxupquote{grover.}}\sphinxbfcode{\sphinxupquote{grover\_run}}}{\emph{target\_state\_vector}, \emph{artificial=False}, \emph{verbose=False}}{}
Runs a simulation of the grover algorithm.
\begin{quote}\begin{description}
\item[{Parameters}] \leavevmode\begin{itemize}
\item {} 
\sphinxstyleliteralstrong{\sphinxupquote{target\_state\_vector}} (\sphinxstyleliteralemphasis{\sphinxupquote{vector}}\sphinxstyleliteralemphasis{\sphinxupquote{{[}}}\sphinxhref{https://docs.python.org/3/library/functions.html\#int}{\sphinxstyleliteralemphasis{\sphinxupquote{int}}}\sphinxstyleliteralemphasis{\sphinxupquote{{]}}}) \textendash{} State searched by Grover’s algorithm 
(only single item searches are supported for now).

\item {} 
\sphinxstyleliteralstrong{\sphinxupquote{artificial}} (\sphinxhref{https://docs.python.org/3/library/functions.html\#bool}{\sphinxstyleliteralemphasis{\sphinxupquote{bool}}}) \textendash{} due to technological limits, it is not always possible 
to compute the states of Grover’s algorithm in a realistic simulator. This 
parameters allows the user to use a non realistic simulator which is way 
quicker.

\item {} 
\sphinxstyleliteralstrong{\sphinxupquote{verbose}} (\sphinxhref{https://docs.python.org/3/library/functions.html\#bool}{\sphinxstyleliteralemphasis{\sphinxupquote{bool}}}) \textendash{} If \(verbose\) then extra run information will be displayed 
in terminal.

\end{itemize}

\item[{Returns}] \leavevmode
list{[}vectors{]} \textendash{} the states at the end of each loop.

\end{description}\end{quote}

\end{fulllineitems}

\index{grover\_vanilla() (in module grover)}

\begin{fulllineitems}
\phantomsection\label{\detokenize{grover:grover.grover_vanilla}}\pysiglinewithargsret{\sphinxcode{\sphinxupquote{grover.}}\sphinxbfcode{\sphinxupquote{grover\_vanilla}}}{\emph{target\_state\_vector}, \emph{verbose=False}}{}
Computes the successive states at the end of the loop using a realistic
simulation of the execution of Grover’s algorithm (using the simulator
from \sphinxcode{\sphinxupquote{run\_circuit.sage}}).
\begin{quote}\begin{description}
\item[{Parameters}] \leavevmode\begin{itemize}
\item {} 
\sphinxstyleliteralstrong{\sphinxupquote{target\_state\_vector}} (\sphinxstyleliteralemphasis{\sphinxupquote{vector}}\sphinxstyleliteralemphasis{\sphinxupquote{{[}}}\sphinxhref{https://docs.python.org/3/library/functions.html\#int}{\sphinxstyleliteralemphasis{\sphinxupquote{int}}}\sphinxstyleliteralemphasis{\sphinxupquote{{]}}}) \textendash{} State searched by Grover’s algorithm 
(only single item searches are supported for now).

\item {} 
\sphinxstyleliteralstrong{\sphinxupquote{verbose}} (\sphinxhref{https://docs.python.org/3/library/functions.html\#bool}{\sphinxstyleliteralemphasis{\sphinxupquote{bool}}}) \textendash{} If \(verbose\) then extra run information will be displayed 
in terminal.

\end{itemize}

\item[{Returns}] \leavevmode
list{[}vector{]} \textendash{} List of states after each application of the 
diffusion operator (as well as the initial state).

\end{description}\end{quote}

\end{fulllineitems}

\index{oracle() (in module grover)}

\begin{fulllineitems}
\phantomsection\label{\detokenize{grover:grover.oracle}}\pysiglinewithargsret{\sphinxcode{\sphinxupquote{grover.}}\sphinxbfcode{\sphinxupquote{oracle}}}{\emph{target\_state\_vector}}{}
The oracle O satisfies \(O|x,y> = |x,f(x)+y>\) where \(f(x)=1\) if \(x\) is the 
element looked for and \(f(x)=0\) otherwise.
\begin{description}
\item[{Example:}] \leavevmode
The oracle flips the qubit where the target is, so :

\fvset{hllines={, ,}}%
\begin{sphinxVerbatim}[commandchars=\\\{\},formatcom=\footnotesize]
\PYG{g+gp}{\PYGZgt{}\PYGZgt{}\PYGZgt{} }\PYG{n}{oracle}\PYG{p}{(}\PYG{n}{target\PYGZus{}state\PYGZus{}ket\PYGZus{}string\PYGZus{}to\PYGZus{}vector}\PYG{p}{(}\PYG{l+s+s2}{\PYGZdq{}}\PYG{l+s+s2}{101}\PYG{l+s+s2}{\PYGZdq{}}\PYG{p}{)}\PYG{p}{)}
\PYG{g+go}{[1 0 0 0 0 0 0 0 0 0 0 0 0 0 0 0]}
\PYG{g+go}{[0 1 0 0 0 0 0 0 0 0 0 0 0 0 0 0]}
\PYG{g+go}{[0 0 1 0 0 0 0 0 0 0 0 0 0 0 0 0]}
\PYG{g+go}{[0 0 0 1 0 0 0 0 0 0 0 0 0 0 0 0]}
\PYG{g+go}{[0 0 0 0 1 0 0 0 0 0 0 0 0 0 0 0]}
\PYG{g+go}{[0 0 0 0 0 1 0 0 0 0 0 0 0 0 0 0]}
\PYG{g+go}{[0 0 0 0 0 0 1 0 0 0 0 0 0 0 0 0]}
\PYG{g+go}{[0 0 0 0 0 0 0 1 0 0 0 0 0 0 0 0]}
\PYG{g+go}{[0 0 0 0 0 0 0 0 1 0 0 0 0 0 0 0]}
\PYG{g+go}{[0 0 0 0 0 0 0 0 0 1 0 0 0 0 0 0]}
\PYG{g+go}{[0 0 0 0 0 0 0 0 0 0 0 1 0 0 0 0]}
\PYG{g+go}{[0 0 0 0 0 0 0 0 0 0 1 0 0 0 0 0]}
\PYG{g+go}{[0 0 0 0 0 0 0 0 0 0 0 0 1 0 0 0]}
\PYG{g+go}{[0 0 0 0 0 0 0 0 0 0 0 0 0 1 0 0]}
\PYG{g+go}{[0 0 0 0 0 0 0 0 0 0 0 0 0 0 1 0]}
\PYG{g+go}{[0 0 0 0 0 0 0 0 0 0 0 0 0 0 0 1]}
\end{sphinxVerbatim}

\end{description}
\begin{quote}\begin{description}
\item[{Parameters}] \leavevmode
\sphinxstyleliteralstrong{\sphinxupquote{target\_state\_vector}} (\sphinxstyleliteralemphasis{\sphinxupquote{vector}}\sphinxstyleliteralemphasis{\sphinxupquote{{[}}}\sphinxhref{https://docs.python.org/3/library/functions.html\#int}{\sphinxstyleliteralemphasis{\sphinxupquote{int}}}\sphinxstyleliteralemphasis{\sphinxupquote{{]}}}) \textendash{} State searched by Grover’s algorithm 
(only single item searches are supported for now).

\item[{Returns}] \leavevmode
matrix \textendash{} The matrix corresponding to the oracle.

\end{description}\end{quote}

\end{fulllineitems}

\index{oracle\_artificial() (in module grover)}

\begin{fulllineitems}
\phantomsection\label{\detokenize{grover:grover.oracle_artificial}}\pysiglinewithargsret{\sphinxcode{\sphinxupquote{grover.}}\sphinxbfcode{\sphinxupquote{oracle\_artificial}}}{\emph{target\_state\_vector}, \emph{V}}{}
Flips the coefficient of \sphinxcode{\sphinxupquote{V}} corresponding to the state 
\sphinxcode{\sphinxupquote{target\_state\_vector}}.
\begin{quote}\begin{description}
\item[{Parameters}] \leavevmode\begin{itemize}
\item {} 
\sphinxstyleliteralstrong{\sphinxupquote{target\_state\_vector}} (\sphinxstyleliteralemphasis{\sphinxupquote{vector}}\sphinxstyleliteralemphasis{\sphinxupquote{{[}}}\sphinxhref{https://docs.python.org/3/library/functions.html\#int}{\sphinxstyleliteralemphasis{\sphinxupquote{int}}}\sphinxstyleliteralemphasis{\sphinxupquote{{]}}}) \textendash{} State searched by Grover’s algorithm 
(only single item searches are supported for now).

\item {} 
\sphinxstyleliteralstrong{\sphinxupquote{V}} (\sphinxstyleliteralemphasis{\sphinxupquote{vector}}) \textendash{} The running state in Grover’s algorithm.

\end{itemize}

\item[{Returns}] \leavevmode
vector \textendash{} \sphinxcode{\sphinxupquote{V}} with the correct coefficient flipped.

\end{description}\end{quote}

\end{fulllineitems}

\index{target\_state\_ket\_list\_to\_vector() (in module grover)}

\begin{fulllineitems}
\phantomsection\label{\detokenize{grover:grover.target_state_ket_list_to_vector}}\pysiglinewithargsret{\sphinxcode{\sphinxupquote{grover.}}\sphinxbfcode{\sphinxupquote{target\_state\_ket\_list\_to\_vector}}}{\emph{target\_state\_ket}}{}
Converts the target state from a digit list to a vector.
\begin{description}
\item[{Example:}] \leavevmode
\(|5>_4 = |0101> = (0, 0, 0, 0, 0, 1, 0, 0, 0, 0, 0, 0, 0, 0, 0, 0)\) so :

\fvset{hllines={, ,}}%
\begin{sphinxVerbatim}[commandchars=\\\{\},formatcom=\footnotesize]
\PYG{g+gp}{\PYGZgt{}\PYGZgt{}\PYGZgt{} }\PYG{n}{target\PYGZus{}state\PYGZus{}ket\PYGZus{}list\PYGZus{}to\PYGZus{}vector}\PYG{p}{(}\PYG{p}{[}\PYG{l+m+mi}{0}\PYG{p}{,}\PYG{l+m+mi}{1}\PYG{p}{,}\PYG{l+m+mi}{0}\PYG{p}{,}\PYG{l+m+mi}{1}\PYG{p}{]}\PYG{p}{)}
\PYG{g+go}{(0, 0, 0, 0, 0, 1, 0, 0, 0, 0, 0, 0, 0, 0, 0, 0)}
\end{sphinxVerbatim}

\end{description}
\begin{quote}\begin{description}
\item[{Parameters}] \leavevmode
\sphinxstyleliteralstrong{\sphinxupquote{target\_state\_ket}} (\sphinxhref{https://docs.python.org/3/library/stdtypes.html\#list}{\sphinxstyleliteralemphasis{\sphinxupquote{list}}}\sphinxstyleliteralemphasis{\sphinxupquote{{[}}}\sphinxhref{https://docs.python.org/3/library/functions.html\#int}{\sphinxstyleliteralemphasis{\sphinxupquote{int}}}\sphinxstyleliteralemphasis{\sphinxupquote{{]}}}) \textendash{} List of digit of the boolean expression of 
the sate looked for.

\item[{Returns}] \leavevmode
vector \textendash{} Vector corresponding to the target state.

\end{description}\end{quote}

\end{fulllineitems}

\index{target\_state\_ket\_string\_to\_vector() (in module grover)}

\begin{fulllineitems}
\phantomsection\label{\detokenize{grover:grover.target_state_ket_string_to_vector}}\pysiglinewithargsret{\sphinxcode{\sphinxupquote{grover.}}\sphinxbfcode{\sphinxupquote{target\_state\_ket\_string\_to\_vector}}}{\emph{target\_state\_ket}}{}
Converts the target state from a string containing a digit list to a vector.
\begin{description}
\item[{Example:}] \leavevmode
\sphinxcode{\sphinxupquote{\textbar{}5\textgreater{}\_4 = \textbar{}0101\textgreater{} = (0, 0, 0, 0, 0, 1, 0, 0, 0, 0, 0, 0, 0, 0, 0, 0)}} so :

\fvset{hllines={, ,}}%
\begin{sphinxVerbatim}[commandchars=\\\{\},formatcom=\footnotesize]
\PYG{g+gp}{\PYGZgt{}\PYGZgt{}\PYGZgt{} }\PYG{n}{target\PYGZus{}state\PYGZus{}ket\PYGZus{}string\PYGZus{}to\PYGZus{}vector}\PYG{p}{(}\PYG{l+s+s2}{\PYGZdq{}}\PYG{l+s+s2}{0101}\PYG{l+s+s2}{\PYGZdq{}}\PYG{p}{)}
\PYG{g+go}{(0, 0, 0, 0, 0, 1, 0, 0, 0, 0, 0, 0, 0, 0, 0, 0)}
\end{sphinxVerbatim}

\end{description}
\begin{quote}\begin{description}
\item[{Parameters}] \leavevmode
\sphinxstyleliteralstrong{\sphinxupquote{target\_state\_ket}} (\sphinxhref{https://docs.python.org/3/library/stdtypes.html\#str}{\sphinxstyleliteralemphasis{\sphinxupquote{str}}}) \textendash{} List of digit of the boolean expression of the 
searched sate.

\item[{Returns}] \leavevmode
vector \textendash{} Vector corresponding to the target state.

\end{description}\end{quote}

\end{fulllineitems}

\index{target\_state\_ket\_vector\_to\_string() (in module grover)}

\begin{fulllineitems}
\phantomsection\label{\detokenize{grover:grover.target_state_ket_vector_to_string}}\pysiglinewithargsret{\sphinxcode{\sphinxupquote{grover.}}\sphinxbfcode{\sphinxupquote{target\_state\_ket\_vector\_to\_string}}}{\emph{target\_state\_ket}}{}
Converts the target state from a vector to a string containing a digit list.
\begin{description}
\item[{Example:}] \leavevmode
\(|5>_4 = |0101> = (0, 0, 0, 0, 0, 1, 0, 0, 0, 0, 0, 0, 0, 0, 0, 0)\) so :

\fvset{hllines={, ,}}%
\begin{sphinxVerbatim}[commandchars=\\\{\},formatcom=\footnotesize]
\PYG{g+gp}{\PYGZgt{}\PYGZgt{}\PYGZgt{} }\PYG{n}{v} \PYG{o}{=} \PYG{n}{vector}\PYG{p}{(}\PYG{p}{(}\PYG{l+m+mi}{0}\PYG{p}{,} \PYG{l+m+mi}{0}\PYG{p}{,} \PYG{l+m+mi}{0}\PYG{p}{,} \PYG{l+m+mi}{0}\PYG{p}{,} \PYG{l+m+mi}{0}\PYG{p}{,} \PYG{l+m+mi}{1}\PYG{p}{,} \PYG{l+m+mi}{0}\PYG{p}{,} \PYG{l+m+mi}{0}\PYG{p}{,} \PYG{l+m+mi}{0}\PYG{p}{,} \PYG{l+m+mi}{0}\PYG{p}{,} \PYG{l+m+mi}{0}\PYG{p}{,} \PYG{l+m+mi}{0}\PYG{p}{,} \PYG{l+m+mi}{0}\PYG{p}{,} \PYG{l+m+mi}{0}\PYG{p}{,} \PYG{l+m+mi}{0}\PYG{p}{,} \PYG{l+m+mi}{0}\PYG{p}{)}\PYG{p}{)}
\PYG{g+gp}{\PYGZgt{}\PYGZgt{}\PYGZgt{} }\PYG{n}{target\PYGZus{}state\PYGZus{}ket\PYGZus{}vector\PYGZus{}to\PYGZus{}string}\PYG{p}{(}\PYG{n}{v}\PYG{p}{)}
\PYG{g+go}{\PYGZsq{}0101\PYGZsq{}}
\end{sphinxVerbatim}

\end{description}
\begin{quote}\begin{description}
\item[{Parameters}] \leavevmode
\sphinxstyleliteralstrong{\sphinxupquote{target\_state\_ket}} (\sphinxstyleliteralemphasis{\sphinxupquote{vector}}) \textendash{} Vector corresponding to the target state.

\item[{Returns}] \leavevmode
string \textendash{} List of digit of the boolean expression of the searched 
sate.

\end{description}\end{quote}

\end{fulllineitems}

\index{time\_step() (in module grover)}

\begin{fulllineitems}
\phantomsection\label{\detokenize{grover:grover.time_step}}\pysiglinewithargsret{\sphinxcode{\sphinxupquote{grover.}}\sphinxbfcode{\sphinxupquote{time\_step}}}{\emph{step\_name}, \emph{step\_function}, \emph{step\_args}, \emph{verbose=False}}{}
Times a function, with time information displayed if \sphinxcode{\sphinxupquote{verbose}}.
\begin{description}
\item[{Example:}] \leavevmode
\fvset{hllines={, ,}}%
\begin{sphinxVerbatim}[commandchars=\\\{\},formatcom=\footnotesize]
\PYG{g+gp}{\PYGZgt{}\PYGZgt{}\PYGZgt{} }\PYG{n}{result} \PYG{o}{=} \PYG{n}{time\PYGZus{}step}\PYG{p}{(}\PYG{l+s+s2}{\PYGZdq{}}\PYG{l+s+s2}{Add\PYGZus{}10}\PYG{l+s+s2}{\PYGZdq{}}\PYG{p}{,} \PYG{k}{lambda} \PYG{n}{a} \PYG{p}{:} \PYG{n}{a} \PYG{o}{+} \PYG{l+m+mi}{10}\PYG{p}{,} \PYG{p}{[}\PYG{l+m+mi}{5}\PYG{p}{]}\PYG{p}{,} \PYG{k+kc}{True}\PYG{p}{)}
\PYG{g+go}{Add\PYGZus{}10 starts now}
\PYG{g+go}{Add\PYGZus{}10 complete, took 1.81198120117e\PYGZhy{}05s}
\PYG{g+gp}{\PYGZgt{}\PYGZgt{}\PYGZgt{} }\PYG{n}{result}
\PYG{g+go}{15}
\end{sphinxVerbatim}

\end{description}
\begin{quote}\begin{description}
\item[{Parameters}] \leavevmode\begin{itemize}
\item {} 
\sphinxstyleliteralstrong{\sphinxupquote{setp\_name}} (\sphinxhref{https://docs.python.org/3/library/stdtypes.html\#str}{\sphinxstyleliteralemphasis{\sphinxupquote{str}}}) \textendash{} Name of the step, used for display.

\item {} 
\sphinxstyleliteralstrong{\sphinxupquote{step\_function}} (\sphinxstyleliteralemphasis{\sphinxupquote{function}}) \textendash{} Function to be timed.

\item {} 
\sphinxstyleliteralstrong{\sphinxupquote{step\_args}} (\sphinxhref{https://docs.python.org/3/library/stdtypes.html\#tuple}{\sphinxstyleliteralemphasis{\sphinxupquote{tuple}}}) \textendash{} Arguments of the function.

\item {} 
\sphinxstyleliteralstrong{\sphinxupquote{verbose}} (\sphinxhref{https://docs.python.org/3/library/functions.html\#bool}{\sphinxstyleliteralemphasis{\sphinxupquote{bool}}}) \textendash{} If \(verbose\) then extra run information will be displayed 
in terminal.

\end{itemize}

\item[{Returns}] \leavevmode
any \textendash{} The output of \sphinxcode{\sphinxupquote{step\_function(*step\_args){}`}}.

\end{description}\end{quote}

\end{fulllineitems}



\section{QFT main}
\label{\detokenize{qft:module-qft}}\label{\detokenize{qft:qft-main}}\label{\detokenize{qft::doc}}\index{qft (module)}
This module is aimed to provide the user means to study the QFT and especially 
how entanglement changes during its execution using the Mermin evaluation.
\index{bit\_value() (in module qft)}

\begin{fulllineitems}
\phantomsection\label{\detokenize{qft:qft.bit_value}}\pysiglinewithargsret{\sphinxcode{\sphinxupquote{qft.}}\sphinxbfcode{\sphinxupquote{bit\_value}}}{\emph{n}, \emph{k}, \emph{base=2}}{}
Returns the value of the \sphinxcode{\sphinxupquote{k}} \(^{th}\) digit of the integer \sphinxcode{\sphinxupquote{n}} given in 
base \sphinxcode{\sphinxupquote{base}}.
\begin{description}
\item[{Example:}] \leavevmode
\fvset{hllines={, ,}}%
\begin{sphinxVerbatim}[commandchars=\\\{\},formatcom=\footnotesize]
\PYG{g+gp}{\PYGZgt{}\PYGZgt{}\PYGZgt{} }\PYG{n}{bit\PYGZus{}value}\PYG{p}{(}\PYG{l+m+mi}{5}\PYG{p}{,} \PYG{l+m+mi}{0}\PYG{p}{)}
\PYG{g+go}{1}
\PYG{g+gp}{\PYGZgt{}\PYGZgt{}\PYGZgt{} }\PYG{n}{bit\PYGZus{}value}\PYG{p}{(}\PYG{l+m+mi}{5}\PYG{p}{,} \PYG{l+m+mi}{1}\PYG{p}{)}
\PYG{g+go}{0}
\PYG{g+gp}{\PYGZgt{}\PYGZgt{}\PYGZgt{} }\PYG{n}{bit\PYGZus{}value}\PYG{p}{(}\PYG{l+m+mi}{15}\PYG{p}{,} \PYG{l+m+mi}{0}\PYG{p}{,} \PYG{n}{base}\PYG{o}{=}\PYG{l+m+mi}{10}\PYG{p}{)}
\PYG{g+go}{5}
\end{sphinxVerbatim}

\end{description}
\begin{quote}\begin{description}
\item[{Parameters}] \leavevmode\begin{itemize}
\item {} 
\sphinxstyleliteralstrong{\sphinxupquote{n}} (\sphinxhref{https://docs.python.org/3/library/functions.html\#int}{\sphinxstyleliteralemphasis{\sphinxupquote{int}}}) \textendash{} Integer studied.

\item {} 
\sphinxstyleliteralstrong{\sphinxupquote{k}} (\sphinxhref{https://docs.python.org/3/library/functions.html\#int}{\sphinxstyleliteralemphasis{\sphinxupquote{int}}}) \textendash{} Digit desired.

\item {} 
\sphinxstyleliteralstrong{\sphinxupquote{base}} (\sphinxhref{https://docs.python.org/3/library/functions.html\#int}{\sphinxstyleliteralemphasis{\sphinxupquote{int}}}) \textendash{} Base in which \sphinxcode{\sphinxupquote{n}} is studied.

\end{itemize}

\item[{Returns}] \leavevmode
int \textendash{} Value of the \sphinxcode{\sphinxupquote{k}} \(^{th}\) digit of the integer \sphinxcode{\sphinxupquote{n}} given 
in base \sphinxcode{\sphinxupquote{base}}.

\end{description}\end{quote}

\end{fulllineitems}

\index{periodic\_state() (in module qft)}

\begin{fulllineitems}
\phantomsection\label{\detokenize{qft:qft.periodic_state}}\pysiglinewithargsret{\sphinxcode{\sphinxupquote{qft.}}\sphinxbfcode{\sphinxupquote{periodic\_state}}}{\emph{l}, \emph{r}, \emph{nWires}}{}
Returns the periodic state \(|\varphi^{l,r}>\) of size \(2^{nWires}\). We have:
\begin{quote}

\(|\varphi^{l,r}> = \sum_{i=0}^{A-1}|l+ir>/sqrt(A)\) with
\(A = floor((2^{nWires}-l)/r)+1\)

In this definition, \sphinxcode{\sphinxupquote{l}} is the shift of the state, and \sphinxcode{\sphinxupquote{r}} is the period 
of the state.
\end{quote}
\begin{description}
\item[{Example:}] \leavevmode
Since
\(|\varphi^{1,5}> = (|1>+|6>+|11>)/sqrt(3)=(|0001>+|0110>+|1011>)/sqrt(3)\),

\fvset{hllines={, ,}}%
\begin{sphinxVerbatim}[commandchars=\\\{\},formatcom=\footnotesize]
\PYG{g+gp}{\PYGZgt{}\PYGZgt{}\PYGZgt{} }\PYG{n}{periodic\PYGZus{}state}\PYG{p}{(}\PYG{l+m+mi}{1}\PYG{p}{,}\PYG{l+m+mi}{5}\PYG{p}{,}\PYG{l+m+mi}{4}\PYG{p}{)}
\PYG{g+go}{(0, 1/3*sqrt(3), 0, 0, 0, 0, 1/3*sqrt(3), 0, 0, 0, 0, 1/3*sqrt(3), 0, 0, 0, 0)}
\end{sphinxVerbatim}

\end{description}
\begin{quote}\begin{description}
\item[{Parameters}] \leavevmode\begin{itemize}
\item {} 
\sphinxstyleliteralstrong{\sphinxupquote{l}} (\sphinxhref{https://docs.python.org/3/library/functions.html\#int}{\sphinxstyleliteralemphasis{\sphinxupquote{int}}}) \textendash{} The shift of the state.

\item {} 
\sphinxstyleliteralstrong{\sphinxupquote{r}} (\sphinxhref{https://docs.python.org/3/library/functions.html\#int}{\sphinxstyleliteralemphasis{\sphinxupquote{int}}}) \textendash{} The period of the state.

\item {} 
\sphinxstyleliteralstrong{\sphinxupquote{nWires}} (\sphinxhref{https://docs.python.org/3/library/functions.html\#int}{\sphinxstyleliteralemphasis{\sphinxupquote{int}}}) \textendash{} The size of the system (number of qubits).

\end{itemize}

\item[{Returns}] \leavevmode
vector \textendash{} The state defined by \sphinxcode{\sphinxupquote{l}}, \sphinxcode{\sphinxupquote{r}} and \sphinxcode{\sphinxupquote{nWires}} according
to the definition given above.

\end{description}\end{quote}

\end{fulllineitems}

\index{qft\_evaluate() (in module qft)}

\begin{fulllineitems}
\phantomsection\label{\detokenize{qft:qft.qft_evaluate}}\pysiglinewithargsret{\sphinxcode{\sphinxupquote{qft.}}\sphinxbfcode{\sphinxupquote{qft\_evaluate}}}{\emph{states}, \emph{verbose=False}, \emph{file\_name=None}}{}
Computes the Mermin Evaluation for each state in \sphinxcode{\sphinxupquote{states}}.
\begin{description}
\item[{Example:}] \leavevmode
\fvset{hllines={, ,}}%
\begin{sphinxVerbatim}[commandchars=\\\{\},formatcom=\footnotesize]
\PYG{g+gp}{\PYGZgt{}\PYGZgt{}\PYGZgt{} }\PYG{n}{qft\PYGZus{}evaluate}\PYG{p}{(}\PYG{p}{[}\PYG{n}{periodic\PYGZus{}state}\PYG{p}{(}\PYG{l+m+mi}{1}\PYG{p}{,}\PYG{l+m+mi}{5}\PYG{p}{,}\PYG{l+m+mi}{4}\PYG{p}{)}\PYG{p}{]}\PYG{p}{)}
\PYG{g+go}{[1.33201398251237]}
\end{sphinxVerbatim}

\end{description}
\begin{quote}\begin{description}
\item[{Parameters}] \leavevmode\begin{itemize}
\item {} 
\sphinxstyleliteralstrong{\sphinxupquote{states}} (\sphinxhref{https://docs.python.org/3/library/stdtypes.html\#list}{\sphinxstyleliteralemphasis{\sphinxupquote{list}}}\sphinxstyleliteralemphasis{\sphinxupquote{{[}}}\sphinxstyleliteralemphasis{\sphinxupquote{vector}}\sphinxstyleliteralemphasis{\sphinxupquote{{]}}}) \textendash{} The states after each step of the QFT.

\item {} 
\sphinxstyleliteralstrong{\sphinxupquote{verbose}} (\sphinxhref{https://docs.python.org/3/library/functions.html\#bool}{\sphinxstyleliteralemphasis{\sphinxupquote{bool}}}) \textendash{} If \(verbose\) then extra run information will be displayed 
in terminal.

\item {} 
\sphinxstyleliteralstrong{\sphinxupquote{file\_name}} (\sphinxhref{https://docs.python.org/3/library/stdtypes.html\#str}{\sphinxstyleliteralemphasis{\sphinxupquote{str}}}) \textendash{} File name for the registration of the Mermin evaluation 
for each step of the algorithm, in csv format.

\end{itemize}

\item[{Returns}] \leavevmode
any \textendash{} Result of this function depends on file\_name. If a file name
is given \(qft_main\) returns None, otherwise, it returns an array with the 
evaluation values.

\end{description}\end{quote}

\end{fulllineitems}

\index{qft\_layers() (in module qft)}

\begin{fulllineitems}
\phantomsection\label{\detokenize{qft:qft.qft_layers}}\pysiglinewithargsret{\sphinxcode{\sphinxupquote{qft.}}\sphinxbfcode{\sphinxupquote{qft\_layers}}}{\emph{state}}{}
Computes the circuit for the QFT using the circuit format  used for the 
\sphinxcode{\sphinxupquote{run}} function from \sphinxcode{\sphinxupquote{run\_circuit.sage}}.
\begin{description}
\item[{Example:}] \leavevmode
\fvset{hllines={, ,}}%
\begin{sphinxVerbatim}[commandchars=\\\{\},formatcom=\footnotesize]
\PYG{g+gp}{\PYGZgt{}\PYGZgt{}\PYGZgt{} }\PYG{n}{qft\PYGZus{}layers}\PYG{p}{(}\PYG{n}{periodic\PYGZus{}state}\PYG{p}{(}\PYG{l+m+mi}{2}\PYG{p}{,}\PYG{l+m+mi}{1}\PYG{p}{,}\PYG{l+m+mi}{3}\PYG{p}{)}\PYG{p}{)}
\PYG{g+go}{[}
\PYG{g+go}{  [}
\PYG{g+go}{    [ 1/2*sqrt(2)  1/2*sqrt(2)]  [1 0]  [1 0]}
\PYG{g+go}{    [ 1/2*sqrt(2) \PYGZhy{}1/2*sqrt(2)], [0 1], [0 1]}
\PYG{g+go}{  ],[}
\PYG{g+go}{    [1 0 0 0 0 0 0 0]}
\PYG{g+go}{    [0 1 0 0 0 0 0 0]}
\PYG{g+go}{    [0 0 1 0 0 0 0 0]}
\PYG{g+go}{    [0 0 0 1 0 0 0 0]}
\PYG{g+go}{    [0 0 0 0 1 0 0 0]}
\PYG{g+go}{    [0 0 0 0 0 1 0 0]}
\PYG{g+go}{    [0 0 0 0 0 0 I 0]}
\PYG{g+go}{    [0 0 0 0 0 0 0 I]}
\PYG{g+go}{  ],[}
\PYG{g+go}{    [1 0 0 0 0                     0 0                     0]}
\PYG{g+go}{    [0 1 0 0 0                     0 0                     0]}
\PYG{g+go}{    [0 0 1 0 0                     0 0                     0]}
\PYG{g+go}{    [0 0 0 1 0                     0 0                     0]}
\PYG{g+go}{    [0 0 0 0 1                     0 0                     0]}
\PYG{g+go}{    [0 0 0 0 0 (1/2*I + 1/2)*sqrt(2) 0                     0]}
\PYG{g+go}{    [0 0 0 0 0                     0 1                     0]}
\PYG{g+go}{    [0 0 0 0 0                     0 0 (1/2*I + 1/2)*sqrt(2)]}
\PYG{g+go}{  ],[}
\PYG{g+go}{    [1 0]  [ 1/2*sqrt(2)  1/2*sqrt(2)]  [1 0]}
\PYG{g+go}{    [0 1], [ 1/2*sqrt(2) \PYGZhy{}1/2*sqrt(2)], [0 1]}
\PYG{g+go}{  ],[}
\PYG{g+go}{    [1 0 0 0 0 0 0 0]}
\PYG{g+go}{    [0 1 0 0 0 0 0 0]}
\PYG{g+go}{    [0 0 1 0 0 0 0 0]}
\PYG{g+go}{    [0 0 0 I 0 0 0 0]}
\PYG{g+go}{    [0 0 0 0 1 0 0 0]}
\PYG{g+go}{    [0 0 0 0 0 1 0 0]}
\PYG{g+go}{    [0 0 0 0 0 0 1 0]}
\PYG{g+go}{    [0 0 0 0 0 0 0 I]}
\PYG{g+go}{  ],[}
\PYG{g+go}{    [1 0]  [1 0]  [ 1/2*sqrt(2)  1/2*sqrt(2)]}
\PYG{g+go}{    [0 1], [0 1], [ 1/2*sqrt(2) \PYGZhy{}1/2*sqrt(2)]}
\PYG{g+go}{  ],[}
\PYG{g+go}{    [1 0 0 0 0 0 0 0]}
\PYG{g+go}{    [0 0 0 0 1 0 0 0]}
\PYG{g+go}{    [0 0 1 0 0 0 0 0]}
\PYG{g+go}{    [0 0 0 0 0 0 1 0]}
\PYG{g+go}{    [0 1 0 0 0 0 0 0]}
\PYG{g+go}{    [0 0 0 0 0 1 0 0]}
\PYG{g+go}{    [0 0 0 1 0 0 0 0]}
\PYG{g+go}{    [0 0 0 0 0 0 0 1]}
\PYG{g+go}{  ]}
\PYG{g+go}{]}
\end{sphinxVerbatim}

\end{description}
\begin{quote}\begin{description}
\item[{Parameters}] \leavevmode
\sphinxstyleliteralstrong{\sphinxupquote{state}} (\sphinxstyleliteralemphasis{\sphinxupquote{vector}}\sphinxstyleliteralemphasis{\sphinxupquote{{[}}}\sphinxhref{https://docs.python.org/3/library/functions.html\#int}{\sphinxstyleliteralemphasis{\sphinxupquote{int}}}\sphinxstyleliteralemphasis{\sphinxupquote{{]}}}) \textendash{} State on which the operator wants the QFT performed 
of, this is usually a periodic state.

\item[{Returns}] \leavevmode
(list{[}list{[}matrix{]}{]},int) \textendash{} Circuit for the QFT.

\end{description}\end{quote}

\end{fulllineitems}

\index{qft\_main() (in module qft)}

\begin{fulllineitems}
\phantomsection\label{\detokenize{qft:qft.qft_main}}\pysiglinewithargsret{\sphinxcode{\sphinxupquote{qft.}}\sphinxbfcode{\sphinxupquote{qft\_main}}}{\emph{state}, \emph{verbose=False}, \emph{file\_name=None}}{}
Prints in terminal or in file \sphinxcode{\sphinxupquote{file\_name}} the Mermin evaluation of each 
step of the QFT.
\begin{description}
\item[{Example:}] \leavevmode
\fvset{hllines={, ,}}%
\begin{sphinxVerbatim}[commandchars=\\\{\},formatcom=\footnotesize]
\PYG{g+gp}{\PYGZgt{}\PYGZgt{}\PYGZgt{} }\PYG{n}{grover}\PYG{p}{(}\PYG{n}{periodic\PYGZus{}state}\PYG{p}{(}\PYG{l+m+mi}{0}\PYG{p}{,}\PYG{l+m+mi}{11}\PYG{p}{,}\PYG{l+m+mi}{4}\PYG{p}{)}\PYG{p}{)}\PYG{l+m+mi}{0}\PYG{p}{,}\PYG{l+m+mf}{1.99823485241887}
\PYG{g+go}{1.99933058720672}
\PYG{g+go}{1.99761683801972}
\PYG{g+go}{1.84600827724524}
\PYG{g+go}{1.66149864543535}
\PYG{g+go}{1.66010335826574}
\PYG{g+go}{1.66127978203391}
\PYG{g+go}{2.17163453049907}
\PYG{g+go}{2.17187381801221}
\PYG{g+go}{1.54230165695219}
\PYG{g+go}{1.54129902140376}
\PYG{g+go}{1.54169705834015}
\end{sphinxVerbatim}

\end{description}
\begin{quote}\begin{description}
\item[{Parameters}] \leavevmode\begin{itemize}
\item {} 
\sphinxstyleliteralstrong{\sphinxupquote{state}} (\sphinxstyleliteralemphasis{\sphinxupquote{vector}}\sphinxstyleliteralemphasis{\sphinxupquote{{[}}}\sphinxhref{https://docs.python.org/3/library/functions.html\#int}{\sphinxstyleliteralemphasis{\sphinxupquote{int}}}\sphinxstyleliteralemphasis{\sphinxupquote{{]}}}) \textendash{} State on which the operator wants the QFT performed 
of, this is usually a periodic state.

\item {} 
\sphinxstyleliteralstrong{\sphinxupquote{verbose}} (\sphinxhref{https://docs.python.org/3/library/functions.html\#bool}{\sphinxstyleliteralemphasis{\sphinxupquote{bool}}}) \textendash{} If \sphinxstyleemphasis{verbose} then extra run information will be displayed 
in terminal.

\item {} 
\sphinxstyleliteralstrong{\sphinxupquote{file\_name}} (\sphinxhref{https://docs.python.org/3/library/stdtypes.html\#str}{\sphinxstyleliteralemphasis{\sphinxupquote{str}}}) \textendash{} File name for the registration of the Mermin evaluation 
for each step of the algorithm, in csv format.

\end{itemize}

\item[{Returns}] \leavevmode
any \textendash{} Result of this function depends on file\_name. If a file name
is given \(qft_main\) returns None, otherwise, it returns an array with the 
evaluation values .

\end{description}\end{quote}

\end{fulllineitems}

\index{qft\_matrix() (in module qft)}

\begin{fulllineitems}
\phantomsection\label{\detokenize{qft:qft.qft_matrix}}\pysiglinewithargsret{\sphinxcode{\sphinxupquote{qft.}}\sphinxbfcode{\sphinxupquote{qft\_matrix}}}{\emph{size}}{}
This function should compute a matrix equivalent to the whole QFT operation. 
It has not been tester much though and should not be relied on for now.
\begin{description}
\item[{Example:}] \leavevmode
\fvset{hllines={, ,}}%
\begin{sphinxVerbatim}[commandchars=\\\{\},formatcom=\footnotesize]
\PYG{g+gp}{\PYGZgt{}\PYGZgt{}\PYGZgt{} }\PYG{n}{qft\PYGZus{}matrix}\PYG{p}{(}\PYG{l+m+mi}{3}\PYG{p}{)}
\PYG{g+go}{[   1/4*sqrt(2)    1/4*sqrt(2)    1/4*sqrt(2)    1/4*sqrt(2)    1/4*sqrt(2)    1/4*sqrt(2)    1/4*sqrt(2)    1/4*sqrt(2)]}
\PYG{g+go}{[   1/4*sqrt(2)    1/4*I + 1/4  1/4*I*sqrt(2)    1/4*I \PYGZhy{} 1/4   \PYGZhy{}1/4*sqrt(2)   \PYGZhy{}1/4*I \PYGZhy{} 1/4 \PYGZhy{}1/4*I*sqrt(2)   \PYGZhy{}1/4*I + 1/4]}
\PYG{g+go}{[   1/4*sqrt(2)  1/4*I*sqrt(2)   \PYGZhy{}1/4*sqrt(2) \PYGZhy{}1/4*I*sqrt(2)    1/4*sqrt(2)  1/4*I*sqrt(2)   \PYGZhy{}1/4*sqrt(2) \PYGZhy{}1/4*I*sqrt(2)]}
\PYG{g+go}{[   1/4*sqrt(2)    1/4*I \PYGZhy{} 1/4 \PYGZhy{}1/4*I*sqrt(2)    1/4*I + 1/4   \PYGZhy{}1/4*sqrt(2)   \PYGZhy{}1/4*I + 1/4  1/4*I*sqrt(2)   \PYGZhy{}1/4*I \PYGZhy{} 1/4]}
\PYG{g+go}{[   1/4*sqrt(2)   \PYGZhy{}1/4*sqrt(2)    1/4*sqrt(2)   \PYGZhy{}1/4*sqrt(2)    1/4*sqrt(2)   \PYGZhy{}1/4*sqrt(2)    1/4*sqrt(2)   \PYGZhy{}1/4*sqrt(2)]}
\PYG{g+go}{[   1/4*sqrt(2)   \PYGZhy{}1/4*I \PYGZhy{} 1/4  1/4*I*sqrt(2)   \PYGZhy{}1/4*I + 1/4   \PYGZhy{}1/4*sqrt(2)    1/4*I + 1/4 \PYGZhy{}1/4*I*sqrt(2)    1/4*I \PYGZhy{} 1/4]}
\PYG{g+go}{[   1/4*sqrt(2) \PYGZhy{}1/4*I*sqrt(2)   \PYGZhy{}1/4*sqrt(2)  1/4*I*sqrt(2)    1/4*sqrt(2) \PYGZhy{}1/4*I*sqrt(2)   \PYGZhy{}1/4*sqrt(2)  1/4*I*sqrt(2)]}
\PYG{g+go}{[   1/4*sqrt(2)   \PYGZhy{}1/4*I + 1/4 \PYGZhy{}1/4*I*sqrt(2)   \PYGZhy{}1/4*I \PYGZhy{} 1/4   \PYGZhy{}1/4*sqrt(2)    1/4*I \PYGZhy{} 1/4  1/4*I*sqrt(2)    1/4*I + 1/4]}
\end{sphinxVerbatim}

\end{description}
\begin{quote}\begin{description}
\item[{Parameters}] \leavevmode
\sphinxstyleliteralstrong{\sphinxupquote{size}} (\sphinxhref{https://docs.python.org/3/library/functions.html\#int}{\sphinxstyleliteralemphasis{\sphinxupquote{int}}}) \textendash{} The size of the QFT (number of qubits).

\item[{Returns}] \leavevmode
matrix \textendash{} The matrix corresponding to the QFT.

\end{description}\end{quote}

\end{fulllineitems}

\index{qft\_run() (in module qft)}

\begin{fulllineitems}
\phantomsection\label{\detokenize{qft:qft.qft_run}}\pysiglinewithargsret{\sphinxcode{\sphinxupquote{qft.}}\sphinxbfcode{\sphinxupquote{qft\_run}}}{\emph{state}, \emph{verbose=False}}{}
Runs a simulation of the QFT.
\begin{description}
\item[{Example:}] \leavevmode
\fvset{hllines={, ,}}%
\begin{sphinxVerbatim}[commandchars=\\\{\},formatcom=\footnotesize]
\PYG{g+gp}{\PYGZgt{}\PYGZgt{}\PYGZgt{} }\PYG{n}{qft\PYGZus{}run}\PYG{p}{(}\PYG{n}{periodic\PYGZus{}state}\PYG{p}{(}\PYG{l+m+mi}{2}\PYG{p}{,}\PYG{l+m+mi}{1}\PYG{p}{,}\PYG{l+m+mi}{3}\PYG{p}{)}\PYG{p}{)}
\PYG{g+go}{[(0, 0, 1/6*sqrt(6), 1/6*sqrt(6), 1/6*sqrt(6), 1/6*sqrt(6), 1/6*sqrt(6), 1/6*sqrt(6)),}
\PYG{g+go}{ (1/12*sqrt(6)*sqrt(2), 1/12*sqrt(6)*sqrt(2), 1/6*sqrt(6)*sqrt(2), 1/6*sqrt(6)*sqrt(2), \PYGZhy{}1/12*sqrt(6)*sqrt(2), \PYGZhy{}1/12*sqrt(6)*sqrt(2), 0, 0),}
\PYG{g+go}{ (1/12*sqrt(6)*sqrt(2), 1/12*sqrt(6)*sqrt(2), 1/6*sqrt(6)*sqrt(2), 1/6*sqrt(6)*sqrt(2), \PYGZhy{}1/12*sqrt(6)*sqrt(2), \PYGZhy{}1/12*sqrt(6)*sqrt(2), 0, 0),}
\PYG{g+go}{ (1/12*sqrt(6)*sqrt(2), 1/12*sqrt(6)*sqrt(2), 1/6*sqrt(6)*sqrt(2), 1/6*sqrt(6)*sqrt(2), \PYGZhy{}1/12*sqrt(6)*sqrt(2), \PYGZhy{}(1/12*I + 1/12)*sqrt(6), 0, 0),}
\PYG{g+go}{ (1/4*sqrt(6), 1/4*sqrt(6), \PYGZhy{}1/12*sqrt(6), \PYGZhy{}1/12*sqrt(6), \PYGZhy{}1/12*sqrt(6), \PYGZhy{}(1/24*I + 1/24)*sqrt(6)*sqrt(2), \PYGZhy{}1/12*sqrt(6), \PYGZhy{}(1/24*I + 1/24)*sqrt(6)*sqrt(2)),}
\PYG{g+go}{ (1/4*sqrt(6), 1/4*sqrt(6), \PYGZhy{}1/12*sqrt(6), \PYGZhy{}1/12*I*sqrt(6), \PYGZhy{}1/12*sqrt(6), \PYGZhy{}(1/24*I + 1/24)*sqrt(6)*sqrt(2), \PYGZhy{}1/12*sqrt(6), \PYGZhy{}(1/24*I \PYGZhy{} 1/24)*sqrt(6)*sqrt(2)),}
\PYG{g+go}{ (1/4*sqrt(6)*sqrt(2), 0, \PYGZhy{}(1/24*I + 1/24)*sqrt(6)*sqrt(2), (1/24*I \PYGZhy{} 1/24)*sqrt(6)*sqrt(2), \PYGZhy{}1/24*sqrt(6)*sqrt(2) \PYGZhy{} (1/24*I + 1/24)*sqrt(6), \PYGZhy{}1/24*sqrt(6)*sqrt(2) + (1/24*I + 1/24)*sqrt(6), \PYGZhy{}1/24*sqrt(6)*sqrt(2) \PYGZhy{} (1/24*I \PYGZhy{} 1/24)*sqrt(6), \PYGZhy{}1/24*sqrt(6)*sqrt(2) + (1/24*I \PYGZhy{} 1/24)*sqrt(6)),}
\PYG{g+go}{ (1/4*sqrt(6)*sqrt(2), \PYGZhy{}1/24*sqrt(6)*sqrt(2) \PYGZhy{} (1/24*I + 1/24)*sqrt(6), \PYGZhy{}(1/24*I + 1/24)*sqrt(6)*sqrt(2), \PYGZhy{}1/24*sqrt(6)*sqrt(2) \PYGZhy{} (1/24*I \PYGZhy{} 1/24)*sqrt(6), 0, \PYGZhy{}1/24*sqrt(6)*sqrt(2) + (1/24*I + 1/24)*sqrt(6), (1/24*I \PYGZhy{} 1/24)*sqrt(6)*sqrt(2), \PYGZhy{}1/24*sqrt(6)*sqrt(2) + (1/24*I \PYGZhy{} 1/24)*sqrt(6))]}
\end{sphinxVerbatim}

\end{description}
\begin{quote}\begin{description}
\item[{Parameters}] \leavevmode\begin{itemize}
\item {} 
\sphinxstyleliteralstrong{\sphinxupquote{state}} (\sphinxstyleliteralemphasis{\sphinxupquote{vector}}\sphinxstyleliteralemphasis{\sphinxupquote{{[}}}\sphinxhref{https://docs.python.org/3/library/functions.html\#int}{\sphinxstyleliteralemphasis{\sphinxupquote{int}}}\sphinxstyleliteralemphasis{\sphinxupquote{{]}}}) \textendash{} State on which the operator wants the QFT performed 
of, this is usually a periodic state.

\item {} 
\sphinxstyleliteralstrong{\sphinxupquote{verbose}} (\sphinxhref{https://docs.python.org/3/library/functions.html\#bool}{\sphinxstyleliteralemphasis{\sphinxupquote{bool}}}) \textendash{} If \sphinxcode{\sphinxupquote{verbose}} then extra run information will be 
displayed in terminal.

\end{itemize}

\item[{Returns}] \leavevmode
list{[}vectors{]} \textendash{} the list of states after each step.

\end{description}\end{quote}

\end{fulllineitems}

\index{set\_bit\_value() (in module qft)}

\begin{fulllineitems}
\phantomsection\label{\detokenize{qft:qft.set_bit_value}}\pysiglinewithargsret{\sphinxcode{\sphinxupquote{qft.}}\sphinxbfcode{\sphinxupquote{set\_bit\_value}}}{\emph{n}, \emph{k}, \emph{value}, \emph{base=2}}{}
Returns \sphinxcode{\sphinxupquote{n}} with its \sphinxcode{\sphinxupquote{k}} \(^{th}\) bit set to \sphinxcode{\sphinxupquote{value}} in base \sphinxcode{\sphinxupquote{base}}.
\begin{description}
\item[{Example:}] \leavevmode
\fvset{hllines={, ,}}%
\begin{sphinxVerbatim}[commandchars=\\\{\},formatcom=\footnotesize]
\PYG{g+gp}{\PYGZgt{}\PYGZgt{}\PYGZgt{} }\PYG{n}{set\PYGZus{}bit\PYGZus{}value}\PYG{p}{(}\PYG{l+m+mi}{5}\PYG{p}{,} \PYG{l+m+mi}{0}\PYG{p}{,} \PYG{l+m+mi}{0}\PYG{p}{)}
\PYG{g+go}{4}
\PYG{g+gp}{\PYGZgt{}\PYGZgt{}\PYGZgt{} }\PYG{n}{set\PYGZus{}bit\PYGZus{}value}\PYG{p}{(}\PYG{l+m+mi}{5}\PYG{p}{,} \PYG{l+m+mi}{1}\PYG{p}{,} \PYG{l+m+mi}{0}\PYG{p}{)}
\PYG{g+go}{5}
\PYG{g+gp}{\PYGZgt{}\PYGZgt{}\PYGZgt{} }\PYG{n}{set\PYGZus{}bit\PYGZus{}value}\PYG{p}{(}\PYG{l+m+mi}{15}\PYG{p}{,} \PYG{l+m+mi}{0}\PYG{p}{,} \PYG{l+m+mi}{9}\PYG{p}{,} \PYG{n}{base}\PYG{o}{=}\PYG{l+m+mi}{10}\PYG{p}{)}
\PYG{g+go}{19}
\end{sphinxVerbatim}

\end{description}
\begin{quote}\begin{description}
\item[{Parameters}] \leavevmode\begin{itemize}
\item {} 
\sphinxstyleliteralstrong{\sphinxupquote{n}} (\sphinxhref{https://docs.python.org/3/library/functions.html\#int}{\sphinxstyleliteralemphasis{\sphinxupquote{int}}}) \textendash{} Integer modified.

\item {} 
\sphinxstyleliteralstrong{\sphinxupquote{k}} (\sphinxhref{https://docs.python.org/3/library/functions.html\#int}{\sphinxstyleliteralemphasis{\sphinxupquote{int}}}) \textendash{} Digit to change.

\item {} 
\sphinxstyleliteralstrong{\sphinxupquote{value}} (\sphinxhref{https://docs.python.org/3/library/functions.html\#int}{\sphinxstyleliteralemphasis{\sphinxupquote{int}}}) \textendash{} Value wanted for the \sphinxcode{\sphinxupquote{k}} \(^{th}\) digit of \sphinxcode{\sphinxupquote{n}} (must be 
between 0 and \sphinxcode{\sphinxupquote{base-1}}).

\item {} 
\sphinxstyleliteralstrong{\sphinxupquote{base}} (\sphinxhref{https://docs.python.org/3/library/functions.html\#int}{\sphinxstyleliteralemphasis{\sphinxupquote{int}}}) \textendash{} Base in which \sphinxcode{\sphinxupquote{n}} is modified.

\end{itemize}

\item[{Returns}] \leavevmode
int \textendash{} \sphinxcode{\sphinxupquote{n}} with its \sphinxcode{\sphinxupquote{k}} \(^{th}\) bit set to \sphinxcode{\sphinxupquote{value}} in base 
\sphinxcode{\sphinxupquote{base}}.

\end{description}\end{quote}

\end{fulllineitems}



\section{Run circuit}
\label{\detokenize{run_circuit:module-run_circuit}}\label{\detokenize{run_circuit:run-circuit}}\label{\detokenize{run_circuit::doc}}\index{run\_circuit (module)}
This module provides a simple quantum circuit simulator.
\index{digit() (in module run\_circuit)}

\begin{fulllineitems}
\phantomsection\label{\detokenize{run_circuit:run_circuit.digit}}\pysiglinewithargsret{\sphinxcode{\sphinxupquote{run\_circuit.}}\sphinxbfcode{\sphinxupquote{digit}}}{\emph{n}, \emph{k}, \emph{base=10}}{}
Computes the digit \sphinxcode{\sphinxupquote{k}} for the integer \sphinxcode{\sphinxupquote{n}} in its representation in base
\sphinxcode{\sphinxupquote{base}}.
\begin{description}
\item[{Example:}] \leavevmode
\fvset{hllines={, ,}}%
\begin{sphinxVerbatim}[commandchars=\\\{\},formatcom=\footnotesize]
\PYG{g+gp}{\PYGZgt{}\PYGZgt{}\PYGZgt{} }\PYG{n}{digit}\PYG{p}{(}\PYG{l+m+mi}{152}\PYG{p}{,} \PYG{l+m+mi}{1}\PYG{p}{)}
\PYG{g+go}{5}
\PYG{g+gp}{\PYGZgt{}\PYGZgt{}\PYGZgt{} }\PYG{n}{digit}\PYG{p}{(}\PYG{l+m+mi}{152}\PYG{p}{,} \PYG{l+m+mi}{0}\PYG{p}{)}
\PYG{g+go}{2}
\end{sphinxVerbatim}

\end{description}
\begin{quote}\begin{description}
\item[{Parameters}] \leavevmode\begin{itemize}
\item {} 
\sphinxstyleliteralstrong{\sphinxupquote{n}} (\sphinxhref{https://docs.python.org/3/library/functions.html\#int}{\sphinxstyleliteralemphasis{\sphinxupquote{int}}}) \textendash{} The integer for which the digit is needed.

\item {} 
\sphinxstyleliteralstrong{\sphinxupquote{k}} (\sphinxhref{https://docs.python.org/3/library/functions.html\#int}{\sphinxstyleliteralemphasis{\sphinxupquote{int}}}) \textendash{} The number of the digit.

\item {} 
\sphinxstyleliteralstrong{\sphinxupquote{base}} (\sphinxhref{https://docs.python.org/3/library/functions.html\#int}{\sphinxstyleliteralemphasis{\sphinxupquote{int}}}) \textendash{} The base used for the representation of \sphinxcode{\sphinxupquote{n}}.

\end{itemize}

\item[{Returns}] \leavevmode
int \textendash{} The \sphinxcode{\sphinxupquote{k}} \(^{th}\) digit of \sphinxcode{\sphinxupquote{n}} in base \sphinxcode{\sphinxupquote{base}}.

\end{description}\end{quote}

\end{fulllineitems}

\index{int\_name() (in module run\_circuit)}

\begin{fulllineitems}
\phantomsection\label{\detokenize{run_circuit:run_circuit.int_name}}\pysiglinewithargsret{\sphinxcode{\sphinxupquote{run\_circuit.}}\sphinxbfcode{\sphinxupquote{int\_name}}}{\emph{num}}{}
Converts a number to a string composed of the list of its digits in English.
\begin{description}
\item[{Example:}] \leavevmode
\fvset{hllines={, ,}}%
\begin{sphinxVerbatim}[commandchars=\\\{\},formatcom=\footnotesize]
\PYG{g+gp}{\PYGZgt{}\PYGZgt{}\PYGZgt{} }\PYG{n}{int\PYGZus{}name}\PYG{p}{(}\PYG{l+m+mi}{152}\PYG{p}{)}
\PYG{g+go}{\PYGZsq{}onefivetwo\PYGZsq{}}
\end{sphinxVerbatim}

\end{description}
\begin{quote}\begin{description}
\item[{Parameters}] \leavevmode
\sphinxstyleliteralstrong{\sphinxupquote{num}} (\sphinxhref{https://docs.python.org/3/library/functions.html\#int}{\sphinxstyleliteralemphasis{\sphinxupquote{int}}}) \textendash{} Number to be converted to a list of digits.

\item[{Returns}] \leavevmode
str \textendash{} List of digits of \sphinxcode{\sphinxupquote{num}} in base 10 concatenated.

\end{description}\end{quote}

\end{fulllineitems}

\index{kronecker() (in module run\_circuit)}

\begin{fulllineitems}
\phantomsection\label{\detokenize{run_circuit:run_circuit.kronecker}}\pysiglinewithargsret{\sphinxcode{\sphinxupquote{run\_circuit.}}\sphinxbfcode{\sphinxupquote{kronecker}}}{\emph{a}, \emph{b}}{}
Computes the Kronecker product of \sphinxcode{\sphinxupquote{a}} and \sphinxcode{\sphinxupquote{b}}.
\begin{description}
\item[{Example:}] \leavevmode
\fvset{hllines={, ,}}%
\begin{sphinxVerbatim}[commandchars=\\\{\},formatcom=\footnotesize]
\PYG{g+gp}{\PYGZgt{}\PYGZgt{}\PYGZgt{} }\PYG{n}{a} \PYG{o}{=} \PYG{n}{matrix}\PYG{p}{(}\PYG{p}{[}\PYG{p}{[}\PYG{l+m+mi}{1}\PYG{p}{,}\PYG{l+m+mi}{0}\PYG{p}{]}\PYG{p}{,}\PYG{p}{[}\PYG{l+m+mi}{0}\PYG{p}{,}\PYG{l+m+mi}{1}\PYG{p}{]}\PYG{p}{]}\PYG{p}{)}
\PYG{g+gp}{\PYGZgt{}\PYGZgt{}\PYGZgt{} }\PYG{n}{b} \PYG{o}{=} \PYG{n}{matrix}\PYG{p}{(}\PYG{p}{[}\PYG{p}{[}\PYG{l+m+mi}{1}\PYG{p}{,}\PYG{l+m+mi}{2}\PYG{p}{]}\PYG{p}{,}\PYG{p}{[}\PYG{l+m+mi}{3}\PYG{p}{,}\PYG{l+m+mi}{4}\PYG{p}{]}\PYG{p}{]}\PYG{p}{)}
\PYG{g+gp}{\PYGZgt{}\PYGZgt{}\PYGZgt{} }\PYG{n}{kronecker}\PYG{p}{(}\PYG{n}{a}\PYG{p}{,}\PYG{n}{b}\PYG{p}{)}
\PYG{g+go}{[1 2 0 0]}
\PYG{g+go}{[3 4 0 0]}
\PYG{g+go}{[0 0 1 2]}
\PYG{g+go}{[0 0 3 4]}
\end{sphinxVerbatim}

\end{description}
\begin{quote}\begin{description}
\item[{Parameters}] \leavevmode
\sphinxstyleliteralstrong{\sphinxupquote{a}}\sphinxstyleliteralstrong{\sphinxupquote{,}}\sphinxstyleliteralstrong{\sphinxupquote{b}} (\sphinxstyleliteralemphasis{\sphinxupquote{matrix}}\sphinxstyleliteralemphasis{\sphinxupquote{,}}\sphinxstyleliteralemphasis{\sphinxupquote{vector}}) \textendash{} Operands for the kronecker operator.

\item[{Returns}] \leavevmode
matrix \sphinxstyleemphasis{or} vector \textendash{}  The kronecker product of \sphinxcode{\sphinxupquote{a}} and \sphinxcode{\sphinxupquote{b}} 
(return type is the same a type of \sphinxcode{\sphinxupquote{a}}).

\end{description}\end{quote}

\end{fulllineitems}

\index{kronecker\_power() (in module run\_circuit)}

\begin{fulllineitems}
\phantomsection\label{\detokenize{run_circuit:run_circuit.kronecker_power}}\pysiglinewithargsret{\sphinxcode{\sphinxupquote{run\_circuit.}}\sphinxbfcode{\sphinxupquote{kronecker\_power}}}{\emph{a}, \emph{n}}{}
Computes the \sphinxcode{\sphinxupquote{n}} \(^{th}\) Kronecker power of \sphinxcode{\sphinxupquote{a}}.
\begin{description}
\item[{Example:}] \leavevmode
\fvset{hllines={, ,}}%
\begin{sphinxVerbatim}[commandchars=\\\{\},formatcom=\footnotesize]
\PYG{g+gp}{\PYGZgt{}\PYGZgt{}\PYGZgt{} }\PYG{n}{a} \PYG{o}{=} \PYG{n}{matrix}\PYG{p}{(}\PYG{p}{[}\PYG{p}{[}\PYG{l+m+mi}{1}\PYG{p}{,}\PYG{l+m+mi}{0}\PYG{p}{]}\PYG{p}{,}\PYG{p}{[}\PYG{l+m+mi}{0}\PYG{p}{,}\PYG{l+m+mi}{2}\PYG{p}{]}\PYG{p}{]}\PYG{p}{)}
\PYG{g+gp}{\PYGZgt{}\PYGZgt{}\PYGZgt{} }\PYG{n}{kronecker\PYGZus{}power}\PYG{p}{(}\PYG{n}{a}\PYG{p}{,}\PYG{l+m+mi}{2}\PYG{p}{)}
\PYG{g+go}{[1 0 0 0]}
\PYG{g+go}{[0 2 0 0]}
\PYG{g+go}{[0 0 2 0]}
\PYG{g+go}{[0 0 0 4]}
\end{sphinxVerbatim}

\end{description}
\begin{quote}\begin{description}
\item[{Parameters}] \leavevmode\begin{itemize}
\item {} 
\sphinxstyleliteralstrong{\sphinxupquote{a}} (\sphinxstyleliteralemphasis{\sphinxupquote{matrix}}\sphinxstyleliteralemphasis{\sphinxupquote{,}}\sphinxstyleliteralemphasis{\sphinxupquote{vector}}) \textendash{} The matrix (or vector) to be elevated to the 
\sphinxcode{\sphinxupquote{n}} \(^{th}\) power.

\item {} 
\sphinxstyleliteralstrong{\sphinxupquote{n}} (\sphinxhref{https://docs.python.org/3/library/functions.html\#int}{\sphinxstyleliteralemphasis{\sphinxupquote{int}}}) \textendash{} The power \sphinxcode{\sphinxupquote{a}} has to be elevated to.

\end{itemize}

\item[{Returns}] \leavevmode
matrix \sphinxstyleemphasis{or} vector \textendash{} The \sphinxcode{\sphinxupquote{n}} \(^{th}\) kronecker power of \sphinxcode{\sphinxupquote{a}} 
(return type is the same a type of \sphinxcode{\sphinxupquote{a}}).

\end{description}\end{quote}

\end{fulllineitems}

\index{layers\_to\_matrix() (in module run\_circuit)}

\begin{fulllineitems}
\phantomsection\label{\detokenize{run_circuit:run_circuit.layers_to_matrix}}\pysiglinewithargsret{\sphinxcode{\sphinxupquote{run\_circuit.}}\sphinxbfcode{\sphinxupquote{layers\_to\_matrix}}}{\emph{layers}}{}
layers is the same as matrix\_layers but with matrix names embedded in them
\begin{description}
\item[{Example:}] \leavevmode
\fvset{hllines={, ,}}%
\begin{sphinxVerbatim}[commandchars=\\\{\},formatcom=\footnotesize]
\PYG{g+gp}{\PYGZgt{}\PYGZgt{}\PYGZgt{} }\PYG{n}{I2} \PYG{o}{=} \PYG{n}{matrix}\PYG{o}{.}\PYG{n}{identity}\PYG{p}{(}\PYG{l+m+mi}{2}\PYG{p}{)}
\PYG{g+gp}{\PYGZgt{}\PYGZgt{}\PYGZgt{} }\PYG{n}{I4} \PYG{o}{=} \PYG{n}{matrix}\PYG{o}{.}\PYG{n}{identity}\PYG{p}{(}\PYG{l+m+mi}{4}\PYG{p}{)}
\PYG{g+gp}{\PYGZgt{}\PYGZgt{}\PYGZgt{} }\PYG{n}{H} \PYG{o}{=} \PYG{n}{matrix}\PYG{p}{(}\PYG{p}{[}\PYG{p}{[}\PYG{l+m+mi}{1}\PYG{p}{,}\PYG{l+m+mi}{1}\PYG{p}{]}\PYG{p}{,}\PYG{p}{[}\PYG{l+m+mi}{1}\PYG{p}{,}\PYG{o}{\PYGZhy{}}\PYG{l+m+mi}{1}\PYG{p}{]}\PYG{p}{]}\PYG{p}{)}\PYG{o}{/}\PYG{n}{sqrt}\PYG{p}{(}\PYG{l+m+mi}{2}\PYG{p}{)}
\PYG{g+gp}{\PYGZgt{}\PYGZgt{}\PYGZgt{} }\PYG{n}{swap} \PYG{o}{=} \PYG{n}{matrix}\PYG{p}{(}\PYG{p}{[}\PYG{p}{[}\PYG{l+m+mi}{1}\PYG{p}{,}\PYG{l+m+mi}{0}\PYG{p}{,}\PYG{l+m+mi}{0}\PYG{p}{,}\PYG{l+m+mi}{0}\PYG{p}{]}\PYG{p}{,}\PYG{p}{[}\PYG{l+m+mi}{0}\PYG{p}{,}\PYG{l+m+mi}{0}\PYG{p}{,}\PYG{l+m+mi}{1}\PYG{p}{,}\PYG{l+m+mi}{0}\PYG{p}{]}\PYG{p}{,}\PYG{p}{[}\PYG{l+m+mi}{0}\PYG{p}{,}\PYG{l+m+mi}{1}\PYG{p}{,}\PYG{l+m+mi}{0}\PYG{p}{,}\PYG{l+m+mi}{0}\PYG{p}{]}\PYG{p}{,}\PYG{p}{[}\PYG{l+m+mi}{0}\PYG{p}{,}\PYG{l+m+mi}{0}\PYG{p}{,}\PYG{l+m+mi}{0}\PYG{p}{,}\PYG{l+m+mi}{1}\PYG{p}{]}\PYG{p}{]}\PYG{p}{)}
\PYG{g+gp}{\PYGZgt{}\PYGZgt{}\PYGZgt{} }\PYG{n}{layers} \PYG{o}{=} \PYG{p}{[}\PYG{p}{[}\PYG{p}{(}\PYG{l+s+s1}{\PYGZsq{}}\PYG{l+s+s1}{I}\PYG{l+s+s1}{\PYGZsq{}}\PYG{p}{,}\PYG{n}{I4}\PYG{p}{)}\PYG{p}{,}\PYG{p}{(}\PYG{l+s+s1}{\PYGZsq{}}\PYG{l+s+s1}{H}\PYG{l+s+s1}{\PYGZsq{}}\PYG{p}{,}\PYG{n}{H}\PYG{p}{)}\PYG{p}{]}\PYG{p}{,}\PYG{p}{[}\PYG{p}{(}\PYG{l+s+s1}{\PYGZsq{}}\PYG{l+s+s1}{H}\PYG{l+s+s1}{\PYGZsq{}}\PYG{p}{,}\PYG{n}{H}\PYG{p}{)}\PYG{p}{,}\PYG{p}{(}\PYG{l+s+s1}{\PYGZsq{}}\PYG{l+s+s1}{H}\PYG{l+s+s1}{\PYGZsq{}}\PYG{p}{,}\PYG{n}{H}\PYG{p}{)}\PYG{p}{,}\PYG{p}{(}\PYG{l+s+s1}{\PYGZsq{}}\PYG{l+s+s1}{I}\PYG{l+s+s1}{\PYGZsq{}}\PYG{p}{,}\PYG{n}{I2}\PYG{p}{)}\PYG{p}{]}\PYG{p}{,}\PYG{p}{[}\PYG{p}{(}\PYG{l+s+s1}{\PYGZsq{}}\PYG{l+s+s1}{I}\PYG{l+s+s1}{\PYGZsq{}}\PYG{p}{,}\PYG{n}{I2}\PYG{p}{)}\PYG{p}{,}\PYG{p}{(}\PYG{l+s+s1}{\PYGZsq{}}\PYG{l+s+s1}{S}\PYG{l+s+s1}{\PYGZsq{}}\PYG{p}{,}\PYG{n}{swap}\PYG{p}{)}\PYG{p}{]}\PYG{p}{]}
\PYG{g+gp}{\PYGZgt{}\PYGZgt{}\PYGZgt{} }\PYG{n}{layers\PYGZus{}to\PYGZus{}matrix}\PYG{p}{(}\PYG{n}{layers}\PYG{p}{)}
\PYG{g+go}{[[I4,H],[H,H,I2],[I2,swap]]}
\end{sphinxVerbatim}

\end{description}
\begin{quote}\begin{description}
\item[{Parameters}] \leavevmode
\sphinxstyleliteralstrong{\sphinxupquote{layers}} (\sphinxhref{https://docs.python.org/3/library/stdtypes.html\#list}{\sphinxstyleliteralemphasis{\sphinxupquote{list}}}\sphinxstyleliteralemphasis{\sphinxupquote{{[}}}\sphinxhref{https://docs.python.org/3/library/stdtypes.html\#list}{\sphinxstyleliteralemphasis{\sphinxupquote{list}}}\sphinxstyleliteralemphasis{\sphinxupquote{{[}}}\sphinxstyleliteralemphasis{\sphinxupquote{(}}\sphinxhref{https://docs.python.org/3/library/stdtypes.html\#str}{\sphinxstyleliteralemphasis{\sphinxupquote{str}}}\sphinxstyleliteralemphasis{\sphinxupquote{,}}\sphinxstyleliteralemphasis{\sphinxupquote{matrix}}\sphinxstyleliteralemphasis{\sphinxupquote{)}}\sphinxstyleliteralemphasis{\sphinxupquote{{]}}}\sphinxstyleliteralemphasis{\sphinxupquote{{]}}}) \textendash{} Circuit under the format described 
above.

\item[{Returns}] \leavevmode
list{[}List{[}(str,int){]}{]} \textendash{} Circuit under the ready-to-run format 
described above.

\end{description}\end{quote}

\end{fulllineitems}

\index{layers\_to\_printable() (in module run\_circuit)}

\begin{fulllineitems}
\phantomsection\label{\detokenize{run_circuit:run_circuit.layers_to_printable}}\pysiglinewithargsret{\sphinxcode{\sphinxupquote{run\_circuit.}}\sphinxbfcode{\sphinxupquote{layers\_to\_printable}}}{\emph{layers}}{}
layers is the same as matrix\_layers but with matrix names embedded in them
\begin{description}
\item[{Example:  }] \leavevmode
\fvset{hllines={, ,}}%
\begin{sphinxVerbatim}[commandchars=\\\{\},formatcom=\footnotesize]
\PYG{g+gp}{\PYGZgt{}\PYGZgt{}\PYGZgt{} }\PYG{n}{I2} \PYG{o}{=} \PYG{n}{matrix}\PYG{o}{.}\PYG{n}{identity}\PYG{p}{(}\PYG{l+m+mi}{2}\PYG{p}{)}
\PYG{g+gp}{\PYGZgt{}\PYGZgt{}\PYGZgt{} }\PYG{n}{I4} \PYG{o}{=} \PYG{n}{matrix}\PYG{o}{.}\PYG{n}{identity}\PYG{p}{(}\PYG{l+m+mi}{4}\PYG{p}{)}
\PYG{g+gp}{\PYGZgt{}\PYGZgt{}\PYGZgt{} }\PYG{n}{H} \PYG{o}{=} \PYG{n}{matrix}\PYG{p}{(}\PYG{p}{[}\PYG{p}{[}\PYG{l+m+mi}{1}\PYG{p}{,}\PYG{l+m+mi}{1}\PYG{p}{]}\PYG{p}{,}\PYG{p}{[}\PYG{l+m+mi}{1}\PYG{p}{,}\PYG{o}{\PYGZhy{}}\PYG{l+m+mi}{1}\PYG{p}{]}\PYG{p}{]}\PYG{p}{)}\PYG{o}{/}\PYG{n}{sqrt}\PYG{p}{(}\PYG{l+m+mi}{2}\PYG{p}{)}
\PYG{g+gp}{\PYGZgt{}\PYGZgt{}\PYGZgt{} }\PYG{n}{swap} \PYG{o}{=} \PYG{n}{matrix}\PYG{p}{(}\PYG{p}{[}\PYG{p}{[}\PYG{l+m+mi}{1}\PYG{p}{,}\PYG{l+m+mi}{0}\PYG{p}{,}\PYG{l+m+mi}{0}\PYG{p}{,}\PYG{l+m+mi}{0}\PYG{p}{]}\PYG{p}{,}\PYG{p}{[}\PYG{l+m+mi}{0}\PYG{p}{,}\PYG{l+m+mi}{0}\PYG{p}{,}\PYG{l+m+mi}{1}\PYG{p}{,}\PYG{l+m+mi}{0}\PYG{p}{]}\PYG{p}{,}\PYG{p}{[}\PYG{l+m+mi}{0}\PYG{p}{,}\PYG{l+m+mi}{1}\PYG{p}{,}\PYG{l+m+mi}{0}\PYG{p}{,}\PYG{l+m+mi}{0}\PYG{p}{]}\PYG{p}{,}\PYG{p}{[}\PYG{l+m+mi}{0}\PYG{p}{,}\PYG{l+m+mi}{0}\PYG{p}{,}\PYG{l+m+mi}{0}\PYG{p}{,}\PYG{l+m+mi}{1}\PYG{p}{]}\PYG{p}{]}\PYG{p}{)}
\PYG{g+gp}{\PYGZgt{}\PYGZgt{}\PYGZgt{} }\PYG{n}{layers} \PYG{o}{=} \PYG{p}{[}\PYG{p}{[}\PYG{p}{(}\PYG{l+s+s1}{\PYGZsq{}}\PYG{l+s+s1}{I}\PYG{l+s+s1}{\PYGZsq{}}\PYG{p}{,}\PYG{n}{I4}\PYG{p}{)}\PYG{p}{,}\PYG{p}{(}\PYG{l+s+s1}{\PYGZsq{}}\PYG{l+s+s1}{H}\PYG{l+s+s1}{\PYGZsq{}}\PYG{p}{,}\PYG{n}{H}\PYG{p}{)}\PYG{p}{]}\PYG{p}{,}\PYG{p}{[}\PYG{p}{(}\PYG{l+s+s1}{\PYGZsq{}}\PYG{l+s+s1}{H}\PYG{l+s+s1}{\PYGZsq{}}\PYG{p}{,}\PYG{n}{H}\PYG{p}{)}\PYG{p}{,}\PYG{p}{(}\PYG{l+s+s1}{\PYGZsq{}}\PYG{l+s+s1}{H}\PYG{l+s+s1}{\PYGZsq{}}\PYG{p}{,}\PYG{n}{H}\PYG{p}{)}\PYG{p}{,}\PYG{p}{(}\PYG{l+s+s1}{\PYGZsq{}}\PYG{l+s+s1}{I}\PYG{l+s+s1}{\PYGZsq{}}\PYG{p}{,}\PYG{n}{I2}\PYG{p}{)}\PYG{p}{]}\PYG{p}{,}\PYG{p}{[}\PYG{p}{(}\PYG{l+s+s1}{\PYGZsq{}}\PYG{l+s+s1}{I}\PYG{l+s+s1}{\PYGZsq{}}\PYG{p}{,}\PYG{n}{I2}\PYG{p}{)}\PYG{p}{,}\PYG{p}{(}\PYG{l+s+s1}{\PYGZsq{}}\PYG{l+s+s1}{S}\PYG{l+s+s1}{\PYGZsq{}}\PYG{p}{,}\PYG{n}{swap}\PYG{p}{)}\PYG{p}{]}\PYG{p}{]}
\PYG{g+go}{[[(\PYGZsq{}I\PYGZsq{},2),(\PYGZsq{}H\PYGZsq{},1)],[(\PYGZsq{}H\PYGZsq{},1),(\PYGZsq{}H\PYGZsq{},1),(\PYGZsq{}I\PYGZsq{},1)],[(\PYGZsq{}I\PYGZsq{},1),(\PYGZsq{}S\PYGZsq{},2)]]}
\end{sphinxVerbatim}

\end{description}
\begin{quote}\begin{description}
\item[{Parameters}] \leavevmode
\sphinxstyleliteralstrong{\sphinxupquote{layers}} (\sphinxhref{https://docs.python.org/3/library/stdtypes.html\#list}{\sphinxstyleliteralemphasis{\sphinxupquote{list}}}\sphinxstyleliteralemphasis{\sphinxupquote{{[}}}\sphinxhref{https://docs.python.org/3/library/stdtypes.html\#list}{\sphinxstyleliteralemphasis{\sphinxupquote{list}}}\sphinxstyleliteralemphasis{\sphinxupquote{{[}}}\sphinxstyleliteralemphasis{\sphinxupquote{(}}\sphinxhref{https://docs.python.org/3/library/stdtypes.html\#str}{\sphinxstyleliteralemphasis{\sphinxupquote{str}}}\sphinxstyleliteralemphasis{\sphinxupquote{,}}\sphinxstyleliteralemphasis{\sphinxupquote{matrix}}\sphinxstyleliteralemphasis{\sphinxupquote{)}}\sphinxstyleliteralemphasis{\sphinxupquote{{]}}}\sphinxstyleliteralemphasis{\sphinxupquote{{]}}}) \textendash{} Circuit under the format described 
above.

\item[{Returns}] \leavevmode
list{[}List{[}(str,int){]}{]} \textendash{} Circuit under the ready-to-print format 
described above.

\end{description}\end{quote}

\end{fulllineitems}

\index{output\_commant() (in module run\_circuit)}

\begin{fulllineitems}
\phantomsection\label{\detokenize{run_circuit:run_circuit.output_commant}}\pysiglinewithargsret{\sphinxcode{\sphinxupquote{run\_circuit.}}\sphinxbfcode{\sphinxupquote{output\_commant}}}{\emph{command\_name}, \emph{command}, \emph{output=False}, \emph{output\_to\_file=False}, \emph{w\_file=None}}{}
This function is used to print useful informations in various ways, see 
arguments details for more information.
\begin{description}
\item[{Example:}] \leavevmode
\fvset{hllines={, ,}}%
\begin{sphinxVerbatim}[commandchars=\\\{\},formatcom=\footnotesize]
\PYG{g+gp}{\PYGZgt{}\PYGZgt{}\PYGZgt{} }\PYG{n}{output\PYGZus{}commant}\PYG{p}{(}\PYG{l+s+s2}{\PYGZdq{}}\PYG{l+s+s2}{test}\PYG{l+s+s2}{\PYGZdq{}}\PYG{p}{,}\PYG{n}{vector}\PYG{p}{(}\PYG{n}{SR}\PYG{p}{,}\PYG{p}{[}\PYG{l+m+mi}{1}\PYG{p}{,}\PYG{l+m+mi}{0}\PYG{p}{,}\PYG{l+m+mi}{0}\PYG{p}{,}\PYG{l+m+mi}{1}\PYG{p}{]}\PYG{p}{)}\PYG{p}{,} \PYG{n}{output}\PYG{o}{=}\PYG{k+kc}{True}\PYG{p}{)}
\PYG{g+go}{\PYGZbs{}newcommand\PYGZob{}\PYGZbs{}test\PYGZcb{}\PYGZob{}1 \PYGZbs{}ket\PYGZob{}00\PYGZcb{} + 1 \PYGZbs{}ket\PYGZob{}11\PYGZcb{}\PYGZcb{}}
\end{sphinxVerbatim}

\end{description}
\begin{quote}\begin{description}
\item[{Parameters}] \leavevmode\begin{itemize}
\item {} 
\sphinxstyleliteralstrong{\sphinxupquote{command\_name}} (\sphinxhref{https://docs.python.org/3/library/stdtypes.html\#str}{\sphinxstyleliteralemphasis{\sphinxupquote{str}}}) \textendash{} The command name.

\item {} 
\sphinxstyleliteralstrong{\sphinxupquote{command}} (\sphinxstyleliteralemphasis{\sphinxupquote{any}}) \textendash{} The command content (if type is \sphinxcode{\sphinxupquote{str}}, will be printed 
as such; if \sphinxcode{\sphinxupquote{vector}}, \sphinxcode{\sphinxupquote{vector\_to\_ket}} will be called on it and if 
\sphinxcode{\sphinxupquote{matrix}}, \sphinxcode{\sphinxupquote{latex}} method from SageMath will be called on it).

\item {} 
\sphinxstyleliteralstrong{\sphinxupquote{output}} (\sphinxhref{https://docs.python.org/3/library/functions.html\#bool}{\sphinxstyleliteralemphasis{\sphinxupquote{bool}}}) \textendash{} disables or enables the output.

\item {} 
\sphinxstyleliteralstrong{\sphinxupquote{output\_to\_file}} (\sphinxhref{https://docs.python.org/3/library/functions.html\#bool}{\sphinxstyleliteralemphasis{\sphinxupquote{bool}}}) \textendash{} Whether the output should be in the standard 
output or in a file.

\item {} 
\sphinxstyleliteralstrong{\sphinxupquote{w\_file}} (\sphinxstyleliteralemphasis{\sphinxupquote{file}}) \textendash{} The opened file to write the output to.

\end{itemize}

\item[{Returns}] \leavevmode
None

\end{description}\end{quote}

\end{fulllineitems}

\index{print\_circuit() (in module run\_circuit)}

\begin{fulllineitems}
\phantomsection\label{\detokenize{run_circuit:run_circuit.print_circuit}}\pysiglinewithargsret{\sphinxcode{\sphinxupquote{run\_circuit.}}\sphinxbfcode{\sphinxupquote{print\_circuit}}}{\emph{name\_layers}, \emph{to\_latex=False}}{}
Each name is a tuple with the name of the gate and its dimension
\begin{description}
\item[{Example:}] \leavevmode
\fvset{hllines={, ,}}%
\begin{sphinxVerbatim}[commandchars=\\\{\},formatcom=\footnotesize]
\PYG{g+gp}{\PYGZgt{}\PYGZgt{}\PYGZgt{} }\PYG{n}{circuit} \PYG{o}{=} \PYG{p}{[}\PYG{p}{[}\PYG{p}{(}\PYG{l+s+s1}{\PYGZsq{}}\PYG{l+s+s1}{I}\PYG{l+s+s1}{\PYGZsq{}}\PYG{p}{,}\PYG{l+m+mi}{2}\PYG{p}{)}\PYG{p}{,}\PYG{p}{(}\PYG{l+s+s1}{\PYGZsq{}}\PYG{l+s+s1}{H}\PYG{l+s+s1}{\PYGZsq{}}\PYG{p}{,}\PYG{l+m+mi}{1}\PYG{p}{)}\PYG{p}{]}\PYG{p}{,}\PYG{p}{[}\PYG{p}{(}\PYG{l+s+s1}{\PYGZsq{}}\PYG{l+s+s1}{H}\PYG{l+s+s1}{\PYGZsq{}}\PYG{p}{,}\PYG{l+m+mi}{1}\PYG{p}{)}\PYG{p}{,}\PYG{p}{(}\PYG{l+s+s1}{\PYGZsq{}}\PYG{l+s+s1}{H}\PYG{l+s+s1}{\PYGZsq{}}\PYG{p}{,}\PYG{l+m+mi}{1}\PYG{p}{)}\PYG{p}{,}\PYG{p}{(}\PYG{l+s+s1}{\PYGZsq{}}\PYG{l+s+s1}{I}\PYG{l+s+s1}{\PYGZsq{}}\PYG{p}{,}\PYG{l+m+mi}{1}\PYG{p}{)}\PYG{p}{]}\PYG{p}{,}\PYG{p}{[}\PYG{p}{(}\PYG{l+s+s1}{\PYGZsq{}}\PYG{l+s+s1}{I}\PYG{l+s+s1}{\PYGZsq{}}\PYG{p}{,}\PYG{l+m+mi}{1}\PYG{p}{)}\PYG{p}{,}\PYG{p}{(}\PYG{l+s+s1}{\PYGZsq{}}\PYG{l+s+s1}{S}\PYG{l+s+s1}{\PYGZsq{}}\PYG{p}{,}\PYG{l+m+mi}{2}\PYG{p}{)}\PYG{p}{]}\PYG{p}{]}
\PYG{g+gp}{\PYGZgt{}\PYGZgt{}\PYGZgt{} }\PYG{n}{print\PYGZus{}circuit}\PYG{p}{(}\PYG{n}{circuit}\PYG{p}{,} \PYG{n}{to\PYGZus{}latex}\PYG{o}{=}\PYG{k+kc}{False}\PYG{p}{)}
\PYG{g+go}{\PYGZhy{}\PYGZhy{}\PYGZhy{}H\PYGZhy{}\PYGZhy{}\PYGZhy{}}
\PYG{g+go}{\PYGZhy{}\PYGZhy{}\PYGZhy{}H\PYGZhy{}S\PYGZhy{}}
\PYG{g+go}{\PYGZhy{}H\PYGZhy{}\PYGZhy{}\PYGZhy{}\textbar{}\PYGZhy{}}
\PYG{g+gp}{\PYGZgt{}\PYGZgt{}\PYGZgt{} }\PYG{n}{print\PYGZus{}circuit}\PYG{p}{(}\PYG{n}{circuit}\PYG{p}{,} \PYG{n}{to\PYGZus{}latex}\PYG{o}{=}\PYG{k+kc}{True}\PYG{p}{)}
\PYG{g+go}{\PYGZbs{}begin\PYGZob{}align*\PYGZcb{}}
\PYG{g+go}{ \PYGZbs{}Qcircuit @C=1em @R=.7em \PYGZob{}}
\PYG{g+go}{  \PYGZam{} \PYGZbs{}qw       \PYGZam{} \PYGZbs{}gate\PYGZob{}H\PYGZcb{}  \PYGZam{} \PYGZbs{}qw               \PYGZam{} \PYGZbs{}qw\PYGZbs{}\PYGZbs{}}
\PYG{g+go}{  \PYGZam{} \PYGZbs{}qw       \PYGZam{} \PYGZbs{}gate\PYGZob{}H\PYGZcb{}  \PYGZam{} \PYGZbs{}multigate\PYGZob{}1\PYGZcb{}\PYGZob{}S\PYGZcb{}  \PYGZam{} \PYGZbs{}qw\PYGZbs{}\PYGZbs{}}
\PYG{g+go}{  \PYGZam{} \PYGZbs{}gate\PYGZob{}H\PYGZcb{}  \PYGZam{} \PYGZbs{}qw       \PYGZam{} \PYGZbs{}ghost\PYGZob{}S\PYGZcb{}         \PYGZam{} \PYGZbs{}qw}
\PYG{g+go}{ \PYGZcb{}}
\PYG{g+go}{\PYGZbs{}end\PYGZob{}align*\PYGZcb{}}
\end{sphinxVerbatim}

\end{description}

\sphinxcode{\sphinxupquote{@multi-gost\_}}, \sphinxcode{\sphinxupquote{@multi-source\_}} and \sphinxcode{\sphinxupquote{@multi-size\_}} are reserved names, 
they should not be used as a gate name.
\begin{quote}\begin{description}
\item[{Parameters}] \leavevmode\begin{itemize}
\item {} 
\sphinxstyleliteralstrong{\sphinxupquote{name\_layers}} (\sphinxhref{https://docs.python.org/3/library/stdtypes.html\#list}{\sphinxstyleliteralemphasis{\sphinxupquote{list}}}\sphinxstyleliteralemphasis{\sphinxupquote{{[}}}\sphinxstyleliteralemphasis{\sphinxupquote{List}}\sphinxstyleliteralemphasis{\sphinxupquote{{[}}}\sphinxstyleliteralemphasis{\sphinxupquote{(}}\sphinxhref{https://docs.python.org/3/library/stdtypes.html\#str}{\sphinxstyleliteralemphasis{\sphinxupquote{str}}}\sphinxstyleliteralemphasis{\sphinxupquote{,}}\sphinxhref{https://docs.python.org/3/library/functions.html\#int}{\sphinxstyleliteralemphasis{\sphinxupquote{int}}}\sphinxstyleliteralemphasis{\sphinxupquote{)}}\sphinxstyleliteralemphasis{\sphinxupquote{{]}}}\sphinxstyleliteralemphasis{\sphinxupquote{{]}}}) \textendash{} Circuit name formalism in the shape 
of the example given above. 
First layer should not be empty.
sum({[}item{[}1{]} for item in layer{]}) should be constant for layer in layers

\item {} 
\sphinxstyleliteralstrong{\sphinxupquote{latex}} (\sphinxhref{https://docs.python.org/3/library/functions.html\#bool}{\sphinxstyleliteralemphasis{\sphinxupquote{bool}}}) \textendash{} If true, a string is returned containing the circuit in
the format given by the LaTeX package \sphinxcode{\sphinxupquote{qcircuit}}. Otherwise, the string
returned is under a custom format meant to be easily readable in the 
terminal.

\end{itemize}

\item[{Returns}] \leavevmode
str \textendash{} The circuit in the format described above.

\end{description}\end{quote}

\end{fulllineitems}

\index{run() (in module run\_circuit)}

\begin{fulllineitems}
\phantomsection\label{\detokenize{run_circuit:run_circuit.run}}\pysiglinewithargsret{\sphinxcode{\sphinxupquote{run\_circuit.}}\sphinxbfcode{\sphinxupquote{run}}}{\emph{matrix\_layers}, \emph{V\_init}, \emph{output=False}, \emph{output\_file=False}, \emph{file=None}, \emph{vector\_name='V'}, \emph{matrix\_name='M'}}{}
Runs the algorithm specified by the matrix\_layers.
\begin{description}
\item[{Example:  }] \leavevmode
\fvset{hllines={, ,}}%
\begin{sphinxVerbatim}[commandchars=\\\{\},formatcom=\footnotesize]
\PYG{g+gp}{\PYGZgt{}\PYGZgt{}\PYGZgt{} }\PYG{n}{I2} \PYG{o}{=} \PYG{n}{matrix}\PYG{o}{.}\PYG{n}{identity}\PYG{p}{(}\PYG{l+m+mi}{2}\PYG{p}{)}
\PYG{g+gp}{\PYGZgt{}\PYGZgt{}\PYGZgt{} }\PYG{n}{I4} \PYG{o}{=} \PYG{n}{matrix}\PYG{o}{.}\PYG{n}{identity}\PYG{p}{(}\PYG{l+m+mi}{4}\PYG{p}{)}
\PYG{g+gp}{\PYGZgt{}\PYGZgt{}\PYGZgt{} }\PYG{n}{H} \PYG{o}{=} \PYG{n}{matrix}\PYG{p}{(}\PYG{p}{[}\PYG{p}{[}\PYG{l+m+mi}{1}\PYG{p}{,}\PYG{l+m+mi}{1}\PYG{p}{]}\PYG{p}{,}\PYG{p}{[}\PYG{l+m+mi}{1}\PYG{p}{,}\PYG{o}{\PYGZhy{}}\PYG{l+m+mi}{1}\PYG{p}{]}\PYG{p}{]}\PYG{p}{)}\PYG{o}{/}\PYG{n}{sqrt}\PYG{p}{(}\PYG{l+m+mi}{2}\PYG{p}{)}
\PYG{g+gp}{\PYGZgt{}\PYGZgt{}\PYGZgt{} }\PYG{n}{swap} \PYG{o}{=} \PYG{n}{matrix}\PYG{p}{(}\PYG{p}{[}\PYG{p}{[}\PYG{l+m+mi}{1}\PYG{p}{,}\PYG{l+m+mi}{0}\PYG{p}{,}\PYG{l+m+mi}{0}\PYG{p}{,}\PYG{l+m+mi}{0}\PYG{p}{]}\PYG{p}{,}\PYG{p}{[}\PYG{l+m+mi}{0}\PYG{p}{,}\PYG{l+m+mi}{0}\PYG{p}{,}\PYG{l+m+mi}{1}\PYG{p}{,}\PYG{l+m+mi}{0}\PYG{p}{]}\PYG{p}{,}\PYG{p}{[}\PYG{l+m+mi}{0}\PYG{p}{,}\PYG{l+m+mi}{1}\PYG{p}{,}\PYG{l+m+mi}{0}\PYG{p}{,}\PYG{l+m+mi}{0}\PYG{p}{]}\PYG{p}{,}\PYG{p}{[}\PYG{l+m+mi}{0}\PYG{p}{,}\PYG{l+m+mi}{0}\PYG{p}{,}\PYG{l+m+mi}{0}\PYG{p}{,}\PYG{l+m+mi}{1}\PYG{p}{]}\PYG{p}{]}\PYG{p}{)}
\PYG{g+gp}{\PYGZgt{}\PYGZgt{}\PYGZgt{} }\PYG{n}{v} \PYG{o}{=} \PYG{n}{vector}\PYG{p}{(}\PYG{p}{[}\PYG{l+m+mi}{1}\PYG{p}{,} \PYG{l+m+mi}{0}\PYG{p}{,} \PYG{l+m+mi}{0}\PYG{p}{,} \PYG{l+m+mi}{0}\PYG{p}{,} \PYG{l+m+mi}{0}\PYG{p}{,} \PYG{l+m+mi}{0}\PYG{p}{,} \PYG{l+m+mi}{0}\PYG{p}{,} \PYG{l+m+mi}{0}\PYG{p}{]}\PYG{p}{)}
\PYG{g+gp}{\PYGZgt{}\PYGZgt{}\PYGZgt{} }\PYG{n}{layers} \PYG{o}{=} \PYG{p}{[}\PYG{p}{[}\PYG{n}{I4}\PYG{p}{,}\PYG{n}{H}\PYG{p}{]}\PYG{p}{,}\PYG{p}{[}\PYG{n}{H}\PYG{p}{,}\PYG{n}{H}\PYG{p}{,}\PYG{n}{I2}\PYG{p}{]}\PYG{p}{,}\PYG{p}{[}\PYG{n}{I2}\PYG{p}{,}\PYG{n}{swap}\PYG{p}{]}\PYG{p}{]}
\PYG{g+gp}{\PYGZgt{}\PYGZgt{}\PYGZgt{} }\PYG{n}{run}\PYG{p}{(}\PYG{n}{layers}\PYG{p}{,}\PYG{n}{v}\PYG{p}{)}
\PYG{g+go}{([(1, 0, 0, 0, 0, 0, 0, 0),}
\PYG{g+go}{  (1/2*sqrt(2), 1/2*sqrt(2), 0, 0, 0, 0, 0, 0),}
\PYG{g+go}{  (1/4*sqrt(2), 1/4*sqrt(2), 1/4*sqrt(2), 1/4*sqrt(2), 1/4*sqrt(2), 1/4*sqrt(2), 1/4*sqrt(2), 1/4*sqrt(2)),}
\PYG{g+go}{  (1/4*sqrt(2), 1/4*sqrt(2), 1/4*sqrt(2), 1/4*sqrt(2), 1/4*sqrt(2), 1/4*sqrt(2), 1/4*sqrt(2), 1/4*sqrt(2))],}
\PYG{g+go}{ [}
\PYG{g+go}{[ 1/2*sqrt(2)  1/2*sqrt(2)            0            0            0            0            0            0]}
\PYG{g+go}{[ 1/2*sqrt(2) \PYGZhy{}1/2*sqrt(2)            0            0            0            0            0            0]}
\PYG{g+go}{[           0            0  1/2*sqrt(2)  1/2*sqrt(2)            0            0            0            0]}
\PYG{g+go}{[           0            0  1/2*sqrt(2) \PYGZhy{}1/2*sqrt(2)            0            0            0            0]}
\PYG{g+go}{[           0            0            0            0  1/2*sqrt(2)  1/2*sqrt(2)            0            0]}
\PYG{g+go}{[           0            0            0            0  1/2*sqrt(2) \PYGZhy{}1/2*sqrt(2)            0            0]}
\PYG{g+go}{[           0            0            0            0            0            0  1/2*sqrt(2)  1/2*sqrt(2)]}
\PYG{g+go}{[           0            0            0            0            0            0  1/2*sqrt(2) \PYGZhy{}1/2*sqrt(2)],}
\PYG{g+go}{[ 1/2    0  1/2    0  1/2    0  1/2    0]  [1 0 0 0 0 0 0 0]}
\PYG{g+go}{[   0  1/2    0  1/2    0  1/2    0  1/2]  [0 0 1 0 0 0 0 0]}
\PYG{g+go}{[ 1/2    0 \PYGZhy{}1/2    0  1/2    0 \PYGZhy{}1/2    0]  [0 1 0 0 0 0 0 0]}
\PYG{g+go}{[   0  1/2    0 \PYGZhy{}1/2    0  1/2    0 \PYGZhy{}1/2]  [0 0 0 1 0 0 0 0]}
\PYG{g+go}{[ 1/2    0  1/2    0 \PYGZhy{}1/2    0 \PYGZhy{}1/2    0]  [0 0 0 0 1 0 0 0]}
\PYG{g+go}{[   0  1/2    0  1/2    0 \PYGZhy{}1/2    0 \PYGZhy{}1/2]  [0 0 0 0 0 0 1 0]}
\PYG{g+go}{[ 1/2    0 \PYGZhy{}1/2    0 \PYGZhy{}1/2    0  1/2    0]  [0 0 0 0 0 1 0 0]}
\PYG{g+go}{[   0  1/2    0 \PYGZhy{}1/2    0 \PYGZhy{}1/2    0  1/2], [0 0 0 0 0 0 0 1]}
\PYG{g+go}{])}
\end{sphinxVerbatim}

\end{description}
\begin{quote}\begin{description}
\item[{Parameters}] \leavevmode\begin{itemize}
\item {} 
\sphinxstyleliteralstrong{\sphinxupquote{matrix\_layers}} (\sphinxstyleliteralemphasis{\sphinxupquote{array}}\sphinxstyleliteralemphasis{\sphinxupquote{{[}}}\sphinxstyleliteralemphasis{\sphinxupquote{array}}\sphinxstyleliteralemphasis{\sphinxupquote{{[}}}\sphinxstyleliteralemphasis{\sphinxupquote{matrix}}\sphinxstyleliteralemphasis{\sphinxupquote{{]}}}\sphinxstyleliteralemphasis{\sphinxupquote{{]}}}) \textendash{} Algorithm described as it would 
be in a circuit.

\item {} 
\sphinxstyleliteralstrong{\sphinxupquote{V\_init}} (\sphinxstyleliteralemphasis{\sphinxupquote{vector}}) \textendash{} Initial input of the algorithm.

\item {} 
\sphinxstyleliteralstrong{\sphinxupquote{output}} (\sphinxhref{https://docs.python.org/3/library/functions.html\#bool}{\sphinxstyleliteralemphasis{\sphinxupquote{bool}}}) \textendash{} If true, outputs are enabled.

\item {} 
\sphinxstyleliteralstrong{\sphinxupquote{output\_to\_file}} (\sphinxhref{https://docs.python.org/3/library/functions.html\#bool}{\sphinxstyleliteralemphasis{\sphinxupquote{bool}}}) \textendash{} If true, trace is returned in file, else it is 
printed.

\item {} 
\sphinxstyleliteralstrong{\sphinxupquote{file}} (\sphinxstyleliteralemphasis{\sphinxupquote{file}}) \textendash{} File to output the commands to.

\item {} 
\sphinxstyleliteralstrong{\sphinxupquote{vector\_name}} (\sphinxhref{https://docs.python.org/3/library/stdtypes.html\#str}{\sphinxstyleliteralemphasis{\sphinxupquote{str}}}) \textendash{} Base name used for the vectors commands.

\item {} 
\sphinxstyleliteralstrong{\sphinxupquote{matrix\_name}} (\sphinxhref{https://docs.python.org/3/library/stdtypes.html\#str}{\sphinxstyleliteralemphasis{\sphinxupquote{str}}}) \textendash{} Base name used for the matrices commands.

\end{itemize}

\item[{Returns}] \leavevmode
vector \textendash{} The states along the execution of the algorithm as well as
the matrix corresponding to each layer.

\end{description}\end{quote}

\end{fulllineitems}



\section{Optimization}
\label{\detokenize{opti:module-opti}}\label{\detokenize{opti:optimization}}\label{\detokenize{opti::doc}}\index{opti (module)}
This module is an simple optimization module that helps find the point of 
maximum value for a given function.
\index{optimize() (in module opti)}

\begin{fulllineitems}
\phantomsection\label{\detokenize{opti:opti.optimize}}\pysiglinewithargsret{\sphinxcode{\sphinxupquote{opti.}}\sphinxbfcode{\sphinxupquote{optimize}}}{\emph{func}, \emph{args\_init}, \emph{step\_init}, \emph{step\_min}, \emph{iter\_max}, \emph{verbose=False}}{}
Optimization function finding the maximum reaching coordinates for \sphinxcode{\sphinxupquote{func}}
with a random walk.
\begin{quote}\begin{description}
\item[{Parameters}] \leavevmode\begin{itemize}
\item {} 
\sphinxstyleliteralstrong{\sphinxupquote{func}} (\sphinxstyleliteralemphasis{\sphinxupquote{function}}) \textendash{} The function to be optimized, each of its arguments must 
be numerical and will be tweaked to find \sphinxcode{\sphinxupquote{func}}’s maximum.

\item {} 
\sphinxstyleliteralstrong{\sphinxupquote{args\_init}} (\sphinxhref{https://docs.python.org/3/library/stdtypes.html\#tuple}{\sphinxstyleliteralemphasis{\sphinxupquote{tuple}}}) \textendash{} Initial coordinates for the random walk.

\item {} 
\sphinxstyleliteralstrong{\sphinxupquote{step\_init}} (\sphinxstyleliteralemphasis{\sphinxupquote{real}}) \textendash{} The step size will vary with time in this function, so
this is the initial value for the step size.

\item {} 
\sphinxstyleliteralstrong{\sphinxupquote{step\_min}} (\sphinxstyleliteralemphasis{\sphinxupquote{real}}) \textendash{} The limit size for the step.

\item {} 
\sphinxstyleliteralstrong{\sphinxupquote{iter\_max}} (\sphinxhref{https://docs.python.org/3/library/functions.html\#int}{\sphinxstyleliteralemphasis{\sphinxupquote{int}}}) \textendash{} Upper iterations bound for each loop to avoid infinite 
loops.

\item {} 
\sphinxstyleliteralstrong{\sphinxupquote{verbose}} (\sphinxhref{https://docs.python.org/3/library/functions.html\#bool}{\sphinxstyleliteralemphasis{\sphinxupquote{bool}}}) \textendash{} If \sphinxcode{\sphinxupquote{verbose}} then extra run information will be 
displayed in terminal.

\end{itemize}

\item[{Returns}] \leavevmode
vector, complex \textendash{} The coordinates of the optimum found for 
\sphinxcode{\sphinxupquote{func}} and the value of \sphinxcode{\sphinxupquote{func}} at this point.

\end{description}\end{quote}

\end{fulllineitems}

\index{optimize\_2spheres() (in module opti)}

\begin{fulllineitems}
\phantomsection\label{\detokenize{opti:opti.optimize_2spheres}}\pysiglinewithargsret{\sphinxcode{\sphinxupquote{opti.}}\sphinxbfcode{\sphinxupquote{optimize\_2spheres}}}{\emph{func}, \emph{args\_init}, \emph{step\_init}, \emph{step\_min}, \emph{iter\_max}, \emph{radius=1}, \emph{verbose=False}}{}
Optimization function finding the maximum reaching coordinates for \sphinxcode{\sphinxupquote{func}}
with a random walk on a two sphere of dimension half the size of 
\sphinxcode{\sphinxupquote{args\_init}}. (Work in progress !)
\begin{quote}

For now, this function is in project and is not used, it can be ignored.
\end{quote}
\begin{quote}\begin{description}
\item[{Parameters}] \leavevmode\begin{itemize}
\item {} 
\sphinxstyleliteralstrong{\sphinxupquote{func}} (\sphinxstyleliteralemphasis{\sphinxupquote{function}}) \textendash{} The function to be optimized, each of its arguments must be 
numerical and will be tweaked to find \sphinxcode{\sphinxupquote{func}}’s maximum.

\item {} 
\sphinxstyleliteralstrong{\sphinxupquote{args\_init}} (\sphinxhref{https://docs.python.org/3/library/stdtypes.html\#tuple}{\sphinxstyleliteralemphasis{\sphinxupquote{tuple}}}) \textendash{} Initial coordinates for the random walk.

\item {} 
\sphinxstyleliteralstrong{\sphinxupquote{step\_init}} (\sphinxstyleliteralemphasis{\sphinxupquote{real}}) \textendash{} The step size will vary with time in this function, so
this is the initial value for the step size.

\item {} 
\sphinxstyleliteralstrong{\sphinxupquote{step\_min}} (\sphinxstyleliteralemphasis{\sphinxupquote{real}}) \textendash{} The limit size for the step.

\item {} 
\sphinxstyleliteralstrong{\sphinxupquote{iter\_max}} (\sphinxhref{https://docs.python.org/3/library/functions.html\#int}{\sphinxstyleliteralemphasis{\sphinxupquote{int}}}) \textendash{} Upper iterations bound for each loop to avoid infinite 
loops.

\item {} 
\sphinxstyleliteralstrong{\sphinxupquote{radius}} (\sphinxstyleliteralemphasis{\sphinxupquote{real}}) \textendash{} Sphere radius.

\item {} 
\sphinxstyleliteralstrong{\sphinxupquote{verbose}} (\sphinxhref{https://docs.python.org/3/library/functions.html\#bool}{\sphinxstyleliteralemphasis{\sphinxupquote{bool}}}) \textendash{} If \sphinxcode{\sphinxupquote{verbose}} then extra run information will be displayed in 
terminal.

\end{itemize}

\item[{Returns}] \leavevmode
vector, complex \textendash{} The coordianates of the optimum found for 
\sphinxcode{\sphinxupquote{func}} and the value of \sphinxcode{\sphinxupquote{func}} at this point.

\end{description}\end{quote}

\end{fulllineitems}

\index{optimize\_normalized() (in module opti)}

\begin{fulllineitems}
\phantomsection\label{\detokenize{opti:opti.optimize_normalized}}\pysiglinewithargsret{\sphinxcode{\sphinxupquote{opti.}}\sphinxbfcode{\sphinxupquote{optimize\_normalized}}}{\emph{func}, \emph{normalizer\_func}, \emph{args\_init}, \emph{step\_init}, \emph{step\_min}, \emph{iter\_max}, \emph{verbose=False}}{}
Optimization function finding the maximum reaching coordinates for \sphinxcode{\sphinxupquote{func}}
with a random walk.
\begin{quote}\begin{description}
\item[{Parameters}] \leavevmode\begin{itemize}
\item {} 
\sphinxstyleliteralstrong{\sphinxupquote{func}} (\sphinxstyleliteralemphasis{\sphinxupquote{function}}) \textendash{} The function to be optimized, each of its arguments must 
be numerical and will be tweaked to find \sphinxcode{\sphinxupquote{func}}’s maximum.

\item {} 
\sphinxstyleliteralstrong{\sphinxupquote{args\_init}} (\sphinxhref{https://docs.python.org/3/library/stdtypes.html\#tuple}{\sphinxstyleliteralemphasis{\sphinxupquote{tuple}}}) \textendash{} Initial coordinates for the random walk.

\item {} 
\sphinxstyleliteralstrong{\sphinxupquote{step\_init}} (\sphinxstyleliteralemphasis{\sphinxupquote{real}}) \textendash{} The step size will vary with time in this function, so
this is the initial value for the step size.

\item {} 
\sphinxstyleliteralstrong{\sphinxupquote{step\_min}} (\sphinxstyleliteralemphasis{\sphinxupquote{real}}) \textendash{} The limit size for the step.

\item {} 
\sphinxstyleliteralstrong{\sphinxupquote{iter\_max}} (\sphinxhref{https://docs.python.org/3/library/functions.html\#int}{\sphinxstyleliteralemphasis{\sphinxupquote{int}}}) \textendash{} Upper iterations bound for each loop to avoid infinite 
loops.

\item {} 
\sphinxstyleliteralstrong{\sphinxupquote{verbose}} (\sphinxhref{https://docs.python.org/3/library/functions.html\#bool}{\sphinxstyleliteralemphasis{\sphinxupquote{bool}}}) \textendash{} If \sphinxcode{\sphinxupquote{verbose}} then extra run information will be 
displayed in terminal.

\end{itemize}

\item[{Returns}] \leavevmode
vector, complex \textendash{} The coordinates of the optimum found for 
\sphinxcode{\sphinxupquote{func}} and the value of \sphinxcode{\sphinxupquote{func}} at this point.

\end{description}\end{quote}

\end{fulllineitems}



\section{Mermin evaluation}
\label{\detokenize{mermin_eval:module-mermin_eval}}\label{\detokenize{mermin_eval:mermin-evaluation}}\label{\detokenize{mermin_eval::doc}}\index{mermin\_eval (module)}
This module is a SageMath module aimed at computing Mermin operators optimized 
to detect a specific quantum state.
\index{M() (in module mermin\_eval)}

\begin{fulllineitems}
\phantomsection\label{\detokenize{mermin_eval:mermin_eval.M}}\pysiglinewithargsret{\sphinxcode{\sphinxupquote{mermin\_eval.}}\sphinxbfcode{\sphinxupquote{M}}}{\emph{n}, \emph{a}, \emph{a\_prime}}{}~\begin{description}
\item[{M\_n is defined as such:}] \leavevmode
M\_n = (1/2)*(M\_(n-1).tensor(a + a’) + M’\_(n-1).tensor(a - a’))

\end{description}
\begin{quote}\begin{description}
\item[{Parameters}] \leavevmode\begin{itemize}
\item {} 
\sphinxstyleliteralstrong{\sphinxupquote{n}} (\sphinxhref{https://docs.python.org/3/library/functions.html\#int}{\sphinxstyleliteralemphasis{\sphinxupquote{int}}}) \textendash{} Iteration for the Mermin operator (determines its size).

\item {} 
\sphinxstyleliteralstrong{\sphinxupquote{a}}\sphinxstyleliteralstrong{\sphinxupquote{,}}\sphinxstyleliteralstrong{\sphinxupquote{a\_prime}} (\sphinxstyleliteralemphasis{\sphinxupquote{matrix}}) \textendash{} Size 2 hermitian operators, defining M as
given above.

\end{itemize}

\item[{Returns}] \leavevmode
matrix \textendash{} A size 2\textasciicircum{}n operator, following the definition given 
above.

\end{description}\end{quote}

\end{fulllineitems}

\index{M\_all() (in module mermin\_eval)}

\begin{fulllineitems}
\phantomsection\label{\detokenize{mermin_eval:mermin_eval.M_all}}\pysiglinewithargsret{\sphinxcode{\sphinxupquote{mermin\_eval.}}\sphinxbfcode{\sphinxupquote{M\_all}}}{\emph{\_n}, \emph{\_a}, \emph{\_a\_prime}}{}~\begin{description}
\item[{M\_n is defined as such:}] \leavevmode
M\_n = (1/2)*(M\_(n-1).tensor(a\_n + a\_n’) + M’\_(n-1).tensor(a-n - a-n’))

\end{description}
\begin{quote}\begin{description}
\item[{Parameters}] \leavevmode\begin{itemize}
\item {} 
\sphinxstyleliteralstrong{\sphinxupquote{n}} (\sphinxhref{https://docs.python.org/3/library/functions.html\#int}{\sphinxstyleliteralemphasis{\sphinxupquote{int}}}) \textendash{} Iteration for the Mermin operator (determines its size).

\item {} 
\sphinxstyleliteralstrong{\sphinxupquote{a}}\sphinxstyleliteralstrong{\sphinxupquote{,}}\sphinxstyleliteralstrong{\sphinxupquote{a\_prime}} (\sphinxhref{https://docs.python.org/3/library/stdtypes.html\#list}{\sphinxstyleliteralemphasis{\sphinxupquote{list}}}\sphinxstyleliteralemphasis{\sphinxupquote{{[}}}\sphinxstyleliteralemphasis{\sphinxupquote{matrix}}\sphinxstyleliteralemphasis{\sphinxupquote{{]}}}) \textendash{} List of size 2 hermitian operators, defining M
as given above.

\end{itemize}

\item[{Returns}] \leavevmode
matrix \textendash{} A size 2\textasciicircum{}n operator, following the definition given 
above.

\end{description}\end{quote}

\end{fulllineitems}

\index{M\_eval() (in module mermin\_eval)}

\begin{fulllineitems}
\phantomsection\label{\detokenize{mermin_eval:mermin_eval.M_eval}}\pysiglinewithargsret{\sphinxcode{\sphinxupquote{mermin\_eval.}}\sphinxbfcode{\sphinxupquote{M\_eval}}}{\emph{a}, \emph{b}, \emph{c}, \emph{m}, \emph{p}, \emph{q}, \emph{phi}}{}
This function evaluates \textless{}\sphinxstyleemphasis{phi*\textbar{}M\_n\textbar{}*phi}\textgreater{} with \sphinxstyleemphasis{(a,b,c,m,p,q)} describing 
M\_n, the Mermin operator.
\begin{quote}

M\_n traditionally uses two families of operators, a\_n and a’\_n, in our 
case, a\_n = a*X+b*Y+c*Z and a’\_n = m*X+p*Y+q*Z.
\end{quote}
\begin{quote}\begin{description}
\item[{Parameters}] \leavevmode\begin{itemize}
\item {} 
\sphinxstyleliteralstrong{\sphinxupquote{a}}\sphinxstyleliteralstrong{\sphinxupquote{,}}\sphinxstyleliteralstrong{\sphinxupquote{b}}\sphinxstyleliteralstrong{\sphinxupquote{,}}\sphinxstyleliteralstrong{\sphinxupquote{c}}\sphinxstyleliteralstrong{\sphinxupquote{,}}\sphinxstyleliteralstrong{\sphinxupquote{m}}\sphinxstyleliteralstrong{\sphinxupquote{,}}\sphinxstyleliteralstrong{\sphinxupquote{p}}\sphinxstyleliteralstrong{\sphinxupquote{,}}\sphinxstyleliteralstrong{\sphinxupquote{q}} (\sphinxstyleliteralemphasis{\sphinxupquote{real}}) \textendash{} Coefficients for the Mermin operator, used as 
described above.

\item {} 
\sphinxstyleliteralstrong{\sphinxupquote{phi}} (\sphinxstyleliteralemphasis{\sphinxupquote{vector}}\sphinxstyleliteralemphasis{\sphinxupquote{{[}}}\sphinxhref{https://docs.python.org/3/library/functions.html\#complex}{\sphinxstyleliteralemphasis{\sphinxupquote{complex}}}\sphinxstyleliteralemphasis{\sphinxupquote{{]}}}) \textendash{} Vector to be evaluated with M.

\end{itemize}

\item[{Returns}] \leavevmode
complex \textendash{} \textless{}\sphinxstyleemphasis{phi*\textbar{}M\_n\textbar{}*phi}\textgreater{}

\end{description}\end{quote}

\end{fulllineitems}

\index{M\_eval\_all() (in module mermin\_eval)}

\begin{fulllineitems}
\phantomsection\label{\detokenize{mermin_eval:mermin_eval.M_eval_all}}\pysiglinewithargsret{\sphinxcode{\sphinxupquote{mermin\_eval.}}\sphinxbfcode{\sphinxupquote{M\_eval\_all}}}{\emph{\_n}, \emph{\_a\_coefs}, \emph{\_a\_prime\_coefs}, \emph{\_rho}}{}
This function evaluates tr(M\_n * rho) with \sphinxstyleemphasis{a} and \sphinxstyleemphasis{a’} describing Mn, the 
Mermin operator.

The coefficients must be given in the following shape:
\textgreater{}\textgreater{}\textgreater{} {[}{[}a,b,c{]}, {[}d,e,f{]}, …{]}
and will result in the following family of observables:
\textgreater{}\textgreater{}\textgreater{} a{[}0{]} = a*X + b*Y + c*Z
\textgreater{}\textgreater{}\textgreater{} a{[}1{]} = d*X + e*Y + f*Z
\textgreater{}\textgreater{}\textgreater{} …
\begin{quote}\begin{description}
\item[{Parameters}] \leavevmode\begin{itemize}
\item {} 
\sphinxstyleliteralstrong{\sphinxupquote{\_n}} (\sphinxhref{https://docs.python.org/3/library/functions.html\#int}{\sphinxstyleliteralemphasis{\sphinxupquote{int}}}) \textendash{} Size of the system.

\item {} 
\sphinxstyleliteralstrong{\sphinxupquote{\_a\_coefs}}\sphinxstyleliteralstrong{\sphinxupquote{, }}\sphinxstyleliteralstrong{\sphinxupquote{\_a\_prime\_coefs}} (\sphinxhref{https://docs.python.org/3/library/stdtypes.html\#list}{\sphinxstyleliteralemphasis{\sphinxupquote{list}}}\sphinxstyleliteralemphasis{\sphinxupquote{{[}}}\sphinxhref{https://docs.python.org/3/library/stdtypes.html\#list}{\sphinxstyleliteralemphasis{\sphinxupquote{list}}}\sphinxstyleliteralemphasis{\sphinxupquote{{[}}}\sphinxstyleliteralemphasis{\sphinxupquote{real}}\sphinxstyleliteralemphasis{\sphinxupquote{{]}}}\sphinxstyleliteralemphasis{\sphinxupquote{{]}}}) \textendash{} Coefficients for the Mermin 
operator, used as described above.

\item {} 
\sphinxstyleliteralstrong{\sphinxupquote{rho}} (\sphinxstyleliteralemphasis{\sphinxupquote{matrix}}\sphinxstyleliteralemphasis{\sphinxupquote{{[}}}\sphinxhref{https://docs.python.org/3/library/functions.html\#complex}{\sphinxstyleliteralemphasis{\sphinxupquote{complex}}}\sphinxstyleliteralemphasis{\sphinxupquote{{]}}}) \textendash{} Density matrix of the state to be evaluated with 
M\_n

\end{itemize}

\item[{Returns}] \leavevmode
complex

\end{description}\end{quote}

\end{fulllineitems}

\index{M\_from\_coef() (in module mermin\_eval)}

\begin{fulllineitems}
\phantomsection\label{\detokenize{mermin_eval:mermin_eval.M_from_coef}}\pysiglinewithargsret{\sphinxcode{\sphinxupquote{mermin\_eval.}}\sphinxbfcode{\sphinxupquote{M\_from\_coef}}}{\emph{n}, \emph{a}, \emph{b}, \emph{c}, \emph{m}, \emph{p}, \emph{q}}{}
Returns the Mermin operator for a given size \sphinxstyleemphasis{n} and coefficients \sphinxstyleemphasis{a} 
through \sphinxstyleemphasis{q}.
\begin{quote}

M traditionally uses two families of operators, a\_n and a’\_n, in our 
case, a\_n = a*X+b*Y+c*Z and a’\_n = m*X+p*Y+q*Z.
\end{quote}
\begin{quote}\begin{description}
\item[{Parameters}] \leavevmode\begin{itemize}
\item {} 
\sphinxstyleliteralstrong{\sphinxupquote{n}} (\sphinxhref{https://docs.python.org/3/library/functions.html\#int}{\sphinxstyleliteralemphasis{\sphinxupquote{int}}}) \textendash{} Iteration for the Mermin operator (determines its size).

\item {} 
\sphinxstyleliteralstrong{\sphinxupquote{a}}\sphinxstyleliteralstrong{\sphinxupquote{,}}\sphinxstyleliteralstrong{\sphinxupquote{b}}\sphinxstyleliteralstrong{\sphinxupquote{,}}\sphinxstyleliteralstrong{\sphinxupquote{c}}\sphinxstyleliteralstrong{\sphinxupquote{,}}\sphinxstyleliteralstrong{\sphinxupquote{m}}\sphinxstyleliteralstrong{\sphinxupquote{,}}\sphinxstyleliteralstrong{\sphinxupquote{p}}\sphinxstyleliteralstrong{\sphinxupquote{,}}\sphinxstyleliteralstrong{\sphinxupquote{q}} (\sphinxstyleliteralemphasis{\sphinxupquote{real}}) \textendash{} Coefficients for the Mermin operator, used as described 
above.

\end{itemize}

\item[{Returns}] \leavevmode
matrix \textendash{} The Mermin operator M\_n.

\end{description}\end{quote}

\end{fulllineitems}

\index{M\_from\_coef\_all() (in module mermin\_eval)}

\begin{fulllineitems}
\phantomsection\label{\detokenize{mermin_eval:mermin_eval.M_from_coef_all}}\pysiglinewithargsret{\sphinxcode{\sphinxupquote{mermin\_eval.}}\sphinxbfcode{\sphinxupquote{M\_from\_coef\_all}}}{\emph{\_n}, \emph{\_a\_coefs}, \emph{\_a\_prime\_coefs}}{}~\begin{description}
\item[{Returns the Mermin operator for a given size \sphinxstyleemphasis{n} and coefficients of \sphinxstyleemphasis{a} }] \leavevmode
and \sphinxstyleemphasis{a’}

\end{description}

The coefficients must be given in the following shape:
\textgreater{}\textgreater{}\textgreater{} {[}{[}a,b,c{]}, {[}d,e,f{]}, …{]}
and will result in the following family of observables:
\textgreater{}\textgreater{}\textgreater{} a{[}0{]} = a*X + b*Y + c*Z
\textgreater{}\textgreater{}\textgreater{} a{[}1{]} = d*X + e*Y + f*Z
\textgreater{}\textgreater{}\textgreater{} …
\begin{quote}\begin{description}
\item[{Parameters}] \leavevmode\begin{itemize}
\item {} 
\sphinxstyleliteralstrong{\sphinxupquote{n}} (\sphinxhref{https://docs.python.org/3/library/functions.html\#int}{\sphinxstyleliteralemphasis{\sphinxupquote{int}}}) \textendash{} Iteration for the Mermin operator (determines its size).

\item {} 
\sphinxstyleliteralstrong{\sphinxupquote{\_a\_coefs}}\sphinxstyleliteralstrong{\sphinxupquote{, }}\sphinxstyleliteralstrong{\sphinxupquote{\_a\_prime\_coefs}} (\sphinxhref{https://docs.python.org/3/library/stdtypes.html\#list}{\sphinxstyleliteralemphasis{\sphinxupquote{list}}}\sphinxstyleliteralemphasis{\sphinxupquote{{[}}}\sphinxhref{https://docs.python.org/3/library/stdtypes.html\#list}{\sphinxstyleliteralemphasis{\sphinxupquote{list}}}\sphinxstyleliteralemphasis{\sphinxupquote{{[}}}\sphinxstyleliteralemphasis{\sphinxupquote{real}}\sphinxstyleliteralemphasis{\sphinxupquote{{]}}}\sphinxstyleliteralemphasis{\sphinxupquote{{]}}}) \textendash{} Coefficients for the Mermin 
operator, used as described above.

\end{itemize}

\item[{Returns}] \leavevmode
matrix \textendash{} The Mermin operator M\_n.

\end{description}\end{quote}

\end{fulllineitems}

\index{M\_prime() (in module mermin\_eval)}

\begin{fulllineitems}
\phantomsection\label{\detokenize{mermin_eval:mermin_eval.M_prime}}\pysiglinewithargsret{\sphinxcode{\sphinxupquote{mermin\_eval.}}\sphinxbfcode{\sphinxupquote{M\_prime}}}{\emph{n}, \emph{a}, \emph{a\_prime}}{}~\begin{description}
\item[{M’\_n is defined as such:}] \leavevmode
M’\_n = (1/2)*(M’\_(n-1).tensor(a + a’) + M\_(n-1).tensor(a’ - a))

\end{description}
\begin{quote}\begin{description}
\item[{Parameters}] \leavevmode\begin{itemize}
\item {} 
\sphinxstyleliteralstrong{\sphinxupquote{n}} (\sphinxhref{https://docs.python.org/3/library/functions.html\#int}{\sphinxstyleliteralemphasis{\sphinxupquote{int}}}) \textendash{} Iteration for the Mermin operator (determines its size).

\item {} 
\sphinxstyleliteralstrong{\sphinxupquote{a}}\sphinxstyleliteralstrong{\sphinxupquote{,}}\sphinxstyleliteralstrong{\sphinxupquote{a\_prime}} (\sphinxstyleliteralemphasis{\sphinxupquote{matrix}}) \textendash{} Size 2 hermitian operators, defining M as
given above.

\end{itemize}

\item[{Returns}] \leavevmode
matrix \textendash{} A size 2\textasciicircum{}n operator, following the definition given 
above.

\end{description}\end{quote}

\end{fulllineitems}

\index{M\_prime\_all() (in module mermin\_eval)}

\begin{fulllineitems}
\phantomsection\label{\detokenize{mermin_eval:mermin_eval.M_prime_all}}\pysiglinewithargsret{\sphinxcode{\sphinxupquote{mermin\_eval.}}\sphinxbfcode{\sphinxupquote{M\_prime\_all}}}{\emph{\_n}, \emph{\_a}, \emph{\_a\_prime}}{}~\begin{description}
\item[{M’\_n is defined as such:}] \leavevmode
M’\_n = (1/2)*(M’\_(n-1).tensor(a\_n + a\_n’) + M\_(n-1).tensor(a\_n’ - a\_n))

\end{description}
\begin{quote}\begin{description}
\item[{Parameters}] \leavevmode\begin{itemize}
\item {} 
\sphinxstyleliteralstrong{\sphinxupquote{n}} (\sphinxhref{https://docs.python.org/3/library/functions.html\#int}{\sphinxstyleliteralemphasis{\sphinxupquote{int}}}) \textendash{} Iteration for the Mermin operator (determines its size).

\item {} 
\sphinxstyleliteralstrong{\sphinxupquote{a}}\sphinxstyleliteralstrong{\sphinxupquote{,}}\sphinxstyleliteralstrong{\sphinxupquote{a\_prime}} (\sphinxhref{https://docs.python.org/3/library/stdtypes.html\#list}{\sphinxstyleliteralemphasis{\sphinxupquote{list}}}\sphinxstyleliteralemphasis{\sphinxupquote{{[}}}\sphinxstyleliteralemphasis{\sphinxupquote{matrix}}\sphinxstyleliteralemphasis{\sphinxupquote{{]}}}) \textendash{} List of size 2 hermitian operators, defining M’ 
as given above.

\end{itemize}

\item[{Returns}] \leavevmode
matrix \textendash{} A size 2\textasciicircum{}n operator, following the definition given 
above.

\end{description}\end{quote}

\end{fulllineitems}

\index{coefficients\_packing() (in module mermin\_eval)}

\begin{fulllineitems}
\phantomsection\label{\detokenize{mermin_eval:mermin_eval.coefficients_packing}}\pysiglinewithargsret{\sphinxcode{\sphinxupquote{mermin\_eval.}}\sphinxbfcode{\sphinxupquote{coefficients\_packing}}}{\emph{\_a\_a\_prime\_coefs}}{}
Packs a list of elements in two lists of lists of three elements

Example:
\textgreater{}\textgreater{}\textgreater{}  coefficients\_packing({[}1,2,3,4,5,6,7,8,9,10,11,12{]})
({[}{[}1,2,3{]},{[}4,5,6{]}{]},{[}{[}7,8,9{]},{[}10,11,12{]}{]})
\begin{description}
\item[{This function is used to interface above \sphinxstyleemphasis{…\_all} functions and the }] \leavevmode
\sphinxstyleemphasis{optimize} function.

\end{description}
\begin{quote}\begin{description}
\item[{Parameters}] \leavevmode
\sphinxstyleliteralstrong{\sphinxupquote{\_a\_a\_prime\_coefs}} (\sphinxhref{https://docs.python.org/3/library/stdtypes.html\#list}{\sphinxstyleliteralemphasis{\sphinxupquote{list}}}\sphinxstyleliteralemphasis{\sphinxupquote{{[}}}\sphinxstyleliteralemphasis{\sphinxupquote{any}}\sphinxstyleliteralemphasis{\sphinxupquote{{]}}}) \textendash{} List of elements.

\item[{Returns}] \leavevmode
tuple{[}list{[}list{[}any{]}{]}{]} \textendash{} Lists of lists of elements as described 
above.

\end{description}\end{quote}

\end{fulllineitems}

\index{coefficients\_unpacking() (in module mermin\_eval)}

\begin{fulllineitems}
\phantomsection\label{\detokenize{mermin_eval:mermin_eval.coefficients_unpacking}}\pysiglinewithargsret{\sphinxcode{\sphinxupquote{mermin\_eval.}}\sphinxbfcode{\sphinxupquote{coefficients\_unpacking}}}{\emph{\_a\_coefs}, \emph{\_a\_prime\_coefs}}{}
Unpacks two lists of lists of three elements to one list of elements

Example:
\textgreater{}\textgreater{}\textgreater{}  coefficients\_unpacking({[}{[}1,2,3{]},{[}4,5,6{]}{]},{[}{[}7,8,9{]},{[}10,11,12{]}{]})
{[}1,2,3,4,5,6,7,8,9,10,11,12{]}
\begin{description}
\item[{This function is used to interface above \sphinxstyleemphasis{…\_all} functions and the }] \leavevmode
\sphinxstyleemphasis{optimize} function.

\end{description}
\begin{quote}\begin{description}
\item[{Parameters}] \leavevmode
\sphinxstyleliteralstrong{\sphinxupquote{\_a\_coefs}}\sphinxstyleliteralstrong{\sphinxupquote{, }}\sphinxstyleliteralstrong{\sphinxupquote{\_a\_prime\_coefs}} (\sphinxhref{https://docs.python.org/3/library/stdtypes.html\#tuple}{\sphinxstyleliteralemphasis{\sphinxupquote{tuple}}}\sphinxstyleliteralemphasis{\sphinxupquote{{[}}}\sphinxhref{https://docs.python.org/3/library/stdtypes.html\#list}{\sphinxstyleliteralemphasis{\sphinxupquote{list}}}\sphinxstyleliteralemphasis{\sphinxupquote{{[}}}\sphinxhref{https://docs.python.org/3/library/stdtypes.html\#list}{\sphinxstyleliteralemphasis{\sphinxupquote{list}}}\sphinxstyleliteralemphasis{\sphinxupquote{{[}}}\sphinxstyleliteralemphasis{\sphinxupquote{any}}\sphinxstyleliteralemphasis{\sphinxupquote{{]}}}\sphinxstyleliteralemphasis{\sphinxupquote{{]}}}\sphinxstyleliteralemphasis{\sphinxupquote{{]}}}) \textendash{} Lists of lists of 
elements as described above.

\item[{Returns}] \leavevmode
list{[}any{]} \textendash{} List of elements.

\end{description}\end{quote}

\end{fulllineitems}

\index{mermin\_coef\_opti() (in module mermin\_eval)}

\begin{fulllineitems}
\phantomsection\label{\detokenize{mermin_eval:mermin_eval.mermin_coef_opti}}\pysiglinewithargsret{\sphinxcode{\sphinxupquote{mermin\_eval.}}\sphinxbfcode{\sphinxupquote{mermin\_coef\_opti}}}{\emph{target\_state}, \emph{verbose=False}}{}
Returns the Mermin operator maximizing the measure for a given input.
\begin{quote}\begin{description}
\item[{Parameters}] \leavevmode\begin{itemize}
\item {} 
\sphinxstyleliteralstrong{\sphinxupquote{target\_state\_vector}} (\sphinxstyleliteralemphasis{\sphinxupquote{vector}}\sphinxstyleliteralemphasis{\sphinxupquote{{[}}}\sphinxhref{https://docs.python.org/3/library/functions.html\#int}{\sphinxstyleliteralemphasis{\sphinxupquote{int}}}\sphinxstyleliteralemphasis{\sphinxupquote{{]}}}) \textendash{} State searched by Grover’s algorithm (only 
single item searches are supported for now).

\item {} 
\sphinxstyleliteralstrong{\sphinxupquote{verbose}} (\sphinxhref{https://docs.python.org/3/library/functions.html\#bool}{\sphinxstyleliteralemphasis{\sphinxupquote{bool}}}) \textendash{} If \sphinxstyleemphasis{verbose} then extra run information will be displayed in 
terminal.

\end{itemize}

\item[{Returns}] \leavevmode
matrix, real \textendash{} Coefficients of the optimal Mermin operator for 
\sphinxstyleemphasis{target\_state} in th Grover algorithm and the value reached.

\end{description}\end{quote}

\end{fulllineitems}

\index{mermin\_coef\_opti\_all() (in module mermin\_eval)}

\begin{fulllineitems}
\phantomsection\label{\detokenize{mermin_eval:mermin_eval.mermin_coef_opti_all}}\pysiglinewithargsret{\sphinxcode{\sphinxupquote{mermin\_eval.}}\sphinxbfcode{\sphinxupquote{mermin\_coef\_opti\_all}}}{\emph{phi}, \emph{verbose=False}}{}
Returns the Mermin operator’ coefficients maximizing tr(M\_n * rho) for a 
given input phi (where \(rho = |phi><phi|\) is the density matrix corresponding
to the state phi).
\begin{quote}\begin{description}
\item[{Parameters}] \leavevmode\begin{itemize}
\item {} 
\sphinxstyleliteralstrong{\sphinxupquote{phi}} (\sphinxstyleliteralemphasis{\sphinxupquote{vector}}\sphinxstyleliteralemphasis{\sphinxupquote{{[}}}\sphinxhref{https://docs.python.org/3/library/functions.html\#complex}{\sphinxstyleliteralemphasis{\sphinxupquote{complex}}}\sphinxstyleliteralemphasis{\sphinxupquote{{]}}}) \textendash{} State used for the optimization of tr(M\_n * rho)
(only single item searches are supported for now).

\item {} 
\sphinxstyleliteralstrong{\sphinxupquote{verbose}} (\sphinxhref{https://docs.python.org/3/library/functions.html\#bool}{\sphinxstyleliteralemphasis{\sphinxupquote{bool}}}) \textendash{} If \sphinxstyleemphasis{verbose} then extra run information will be displayed in 
terminal.

\end{itemize}

\item[{Returns}] \leavevmode
list{[}real{]}, real \textendash{} Coefficients of the optimal Mermin operator for 
\sphinxstyleemphasis{target\_state} in th Grover algorithm and the value reached.

\end{description}\end{quote}

\end{fulllineitems}

\index{mermin\_operator\_opti() (in module mermin\_eval)}

\begin{fulllineitems}
\phantomsection\label{\detokenize{mermin_eval:mermin_eval.mermin_operator_opti}}\pysiglinewithargsret{\sphinxcode{\sphinxupquote{mermin\_eval.}}\sphinxbfcode{\sphinxupquote{mermin\_operator\_opti}}}{\emph{target\_state\_vector}, \emph{precomputed\_filename=None}, \emph{verbose=False}}{}
Computes the pseudo optimal operator used to perform the Mermin evaluation 
during Grover’s algorithm.
\begin{quote}\begin{description}
\item[{Parameters}] \leavevmode\begin{itemize}
\item {} 
\sphinxstyleliteralstrong{\sphinxupquote{target\_state\_vector}} (\sphinxstyleliteralemphasis{\sphinxupquote{vector}}\sphinxstyleliteralemphasis{\sphinxupquote{{[}}}\sphinxhref{https://docs.python.org/3/library/functions.html\#int}{\sphinxstyleliteralemphasis{\sphinxupquote{int}}}\sphinxstyleliteralemphasis{\sphinxupquote{{]}}}) \textendash{} State searched by Grover’s algorithm (only 
single item searches are supported for now).

\item {} 
\sphinxstyleliteralstrong{\sphinxupquote{precomputed\_filename}} (\sphinxhref{https://docs.python.org/3/library/stdtypes.html\#str}{\sphinxstyleliteralemphasis{\sphinxupquote{str}}}) \textendash{} File where the precomputed coefficients for 
optimal Mermin operator are stored, if left empty,  precomputation will
not be used. If precomputation is used and the searched state is not in 
this database, once the optimization done, the result will be added to 
the file.

\item {} 
\sphinxstyleliteralstrong{\sphinxupquote{verbose}} (\sphinxhref{https://docs.python.org/3/library/functions.html\#bool}{\sphinxstyleliteralemphasis{\sphinxupquote{bool}}}) \textendash{} If \sphinxstyleemphasis{verbose} then extra run information will be displayed in 
terminal.

\end{itemize}

\item[{Returns}] \leavevmode
matrix, real \textendash{} The Mermin operator satisfying the required conditions
and the value reached.

\end{description}\end{quote}

\end{fulllineitems}



\section{Vector to Ket}
\label{\detokenize{vector_to_ket:module-vector_to_ket}}\label{\detokenize{vector_to_ket:vector-to-ket}}\label{\detokenize{vector_to_ket::doc}}\index{vector\_to\_ket (module)}
This module is an ease of life adding to SageMath: it complements the \sphinxcode{\sphinxupquote{latex}}
method to quantum ket notations.
\index{int\_index\_to\_ket() (in module vector\_to\_ket)}

\begin{fulllineitems}
\phantomsection\label{\detokenize{vector_to_ket:vector_to_ket.int_index_to_ket}}\pysiglinewithargsret{\sphinxcode{\sphinxupquote{vector\_to\_ket.}}\sphinxbfcode{\sphinxupquote{int\_index\_to\_ket}}}{\emph{index}, \emph{register\_size}}{}
Makes a int from a register into a binary ket notation.
IMPORTANT: this notation requires the \sphinxstyleemphasis{physics} package in Latex
\begin{description}
\item[{Example:}] \leavevmode
\fvset{hllines={, ,}}%
\begin{sphinxVerbatim}[commandchars=\\\{\},formatcom=\footnotesize]
\PYG{g+gp}{\PYGZgt{}\PYGZgt{}\PYGZgt{} }\PYG{n}{int\PYGZus{}index\PYGZus{}to\PYGZus{}ket}\PYG{p}{(}\PYG{l+m+mi}{2}\PYG{p}{,}\PYG{l+m+mi}{2}\PYG{p}{)}
\PYG{g+go}{\PYGZsq{}\PYGZbs{}\PYGZbs{}ket\PYGZob{}10\PYGZcb{}\PYGZsq{}}
\end{sphinxVerbatim}

\end{description}
\begin{quote}\begin{description}
\item[{Parameters}] \leavevmode\begin{itemize}
\item {} 
\sphinxstyleliteralstrong{\sphinxupquote{index}} (\sphinxhref{https://docs.python.org/3/library/functions.html\#int}{\sphinxstyleliteralemphasis{\sphinxupquote{int}}}) \textendash{} the index to be transformed

\item {} 
\sphinxstyleliteralstrong{\sphinxupquote{register\_size}} (\sphinxhref{https://docs.python.org/3/library/functions.html\#int}{\sphinxstyleliteralemphasis{\sphinxupquote{int}}}) \textendash{} the number of qubits in the system

\end{itemize}

\item[{Returns}] \leavevmode
string \textendash{} a string in latex format

\end{description}\end{quote}

\end{fulllineitems}

\index{vector\_to\_ket() (in module vector\_to\_ket)}

\begin{fulllineitems}
\phantomsection\label{\detokenize{vector_to_ket:vector_to_ket.vector_to_ket}}\pysiglinewithargsret{\sphinxcode{\sphinxupquote{vector\_to\_ket.}}\sphinxbfcode{\sphinxupquote{vector\_to\_ket}}}{\emph{v}}{}
Transform a sage vector to a Latex ket notation
\begin{description}
\item[{Example:}] \leavevmode
\fvset{hllines={, ,}}%
\begin{sphinxVerbatim}[commandchars=\\\{\},formatcom=\footnotesize]
\PYG{g+gp}{\PYGZgt{}\PYGZgt{}\PYGZgt{} }\PYG{n}{v} \PYG{o}{=} \PYG{n}{vector}\PYG{p}{(}\PYG{n}{SR}\PYG{p}{,}\PYG{p}{[}\PYG{l+m+mi}{1}\PYG{p}{,}\PYG{l+m+mi}{0}\PYG{p}{,}\PYG{l+m+mi}{0}\PYG{p}{,}\PYG{l+m+mi}{1}\PYG{p}{]}\PYG{p}{)}
\PYG{g+gp}{\PYGZgt{}\PYGZgt{}\PYGZgt{} }\PYG{n}{vector\PYGZus{}to\PYGZus{}ket}\PYG{p}{(}\PYG{n}{v}\PYG{p}{)}
\PYG{g+go}{1 \PYGZbs{}ket\PYGZob{}00\PYGZcb{} + 1 \PYGZbs{}ket\PYGZob{}11\PYGZcb{}}
\end{sphinxVerbatim}

\end{description}
\begin{quote}\begin{description}
\item[{Parameters}] \leavevmode
\sphinxstyleliteralstrong{\sphinxupquote{v}} (\sphinxstyleliteralemphasis{\sphinxupquote{vector}}) \textendash{} the vector corresponding to a pure state (size must be a power of 
two)

\item[{Returns}] \leavevmode
str \textendash{} a string corresponding to the Latex code for the ket 
notation of the input vector

\end{description}\end{quote}

\end{fulllineitems}



\chapter{Indices and Tables}
\label{\detokenize{index:indices-and-tables}}\begin{itemize}
\item {} 
\DUrole{xref,std,std-ref}{genindex}

\item {} 
\DUrole{xref,std,std-ref}{modindex}

\item {} 
\DUrole{xref,std,std-ref}{search}

\end{itemize}


\renewcommand{\indexname}{Python Module Index}
\begin{sphinxtheindex}
\def\bigletter#1{{\Large\sffamily#1}\nopagebreak\vspace{1mm}}
\bigletter{g}
\item {\sphinxstyleindexentry{grover}}\sphinxstyleindexpageref{grover:\detokenize{module-grover}}
\indexspace
\bigletter{m}
\item {\sphinxstyleindexentry{mermin\_eval}}\sphinxstyleindexpageref{mermin_eval:\detokenize{module-mermin_eval}}
\indexspace
\bigletter{o}
\item {\sphinxstyleindexentry{opti}}\sphinxstyleindexpageref{opti:\detokenize{module-opti}}
\indexspace
\bigletter{q}
\item {\sphinxstyleindexentry{qft}}\sphinxstyleindexpageref{qft:\detokenize{module-qft}}
\indexspace
\bigletter{r}
\item {\sphinxstyleindexentry{run\_circuit}}\sphinxstyleindexpageref{run_circuit:\detokenize{module-run_circuit}}
\indexspace
\bigletter{v}
\item {\sphinxstyleindexentry{vector\_to\_ket}}\sphinxstyleindexpageref{vector_to_ket:\detokenize{module-vector_to_ket}}
\end{sphinxtheindex}

\renewcommand{\indexname}{Index}
\printindex
\end{document}